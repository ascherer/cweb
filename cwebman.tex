\input cwebmac
\parskip 0pt plus 1pt
\def\RA{\char'31 } % right arrow
\def\hang{\hangindent 4em\ignorespaces}
\font\eightrm=cmr8
\font\ninerm=cmr9
\font\ninett=cmtt9
\font\eighttt=cmtt8
\def\Pascal{Pascal}
\font\quoterm=cmssq8
\font\quoteit=cmssqi8
\def\pb{\.{|...|}}
\def\v{\.{\char'174}} % vertical (|) in typewriter font
\def\lpile{\def\cr{\hfill\endline}\matrix} % I only use \lpile by itself
\abovedisplayskip=.5\abovedisplayskip
\belowdisplayskip=.5\belowdisplayskip
\abovedisplayshortskip=.5\abovedisplayshortskip
\belowdisplayshortskip=.5\belowdisplayshortskip
\advance\pageheight by \baselineskip % the manual just got a bit longer
\advance\fullpageheight by \baselineskip
\setpage

\outer\def\section #1.{\penalty-50\vskip 6pt plus 3pt minus 3pt
  \noindent{\bf #1.}\quad\ignorespaces}

\def\lheader{\mainfont\the\pageno\hfill\sc\runninghead\hfill}
\def\rheader{\hfill\sc\runninghead\hfill\mainfont\the\pageno}
\def\runninghead{{\tentt CWEB} USER MANUAL (VERSION 2.8)}

% This verbatim mode assumes that ! marks are !! in the text being copied.
\def\verbatim{\begingroup
  \def\do##1{\catcode`##1=12 } \dospecials
  \parskip 0pt \parindent 0pt \let\!=!
  \catcode`\ =13 \catcode`\^^M=13
  \tt \catcode`\!=0 \verbatimdefs \verbatimgobble}
{\catcode`\^^M=13{\catcode`\ =13\gdef\verbatimdefs{\def^^M{\ \par}\let =\ }} %
  \gdef\verbatimgobble#1^^M{}}

\null\vfill
\noindent
Copyright \copyright\ 1987, 1990, 1992 Silvio Levy and Donald E. Knuth
\bigskip\noindent
Permission is granted to make and distribute verbatim copies of this
document provided that the copyright notice and this permission notice
are preserved on all copies.

\smallskip\noindent
Permission is granted to copy and distribute modified versions of this
document under the conditions for verbatim copying, provided that the
entire resulting derived work is distributed under the terms of a
permission notice identical to this one.

\pageno=0 \titletrue\eject

\centerline{\titlefont The {\ttitlefont CWEB} System of
    Structured Documentation}

\vskip 15pt plus 3pt minus 3pt
\noindent
This document describes a version of Don Knuth's \.{WEB} system,
adapted to \Cee\ by Silvio Levy.  Knuth's original Pascal programs have been
entirely rewritten in \Cee; many changes were made to take
advantage of features offered by \Cee\ but non-existent in \Pascal.

Readers who are familiar with Knuth's memo ``The \.{WEB} System of Structured
Documentation'' will be able
to skim this material rapidly, because \.{CWEB} is essentially a
simplified subset of \.{WEB}. \.{CWEB} does not need \.{WEB}'s features
for macro definition and string handling, because \Cee\ and its
preprocessor already take care of macros and strings. Similarly, the \.{WEB}
conventions of denoting octal and hexadecimal constants by \.{@'77}
and \.{@"3f} are replaced by \Cee's conventions \.{\v077\v} and
\.{\v0x3f\v}, respectively. All other features of \.{WEB} have been
retained, and a few new features have been added.

\section Introduction.
The philosophy behind \.{WEB} is
that an experienced system programmer, who wants to provide the best
possible documentation of software products, needs two things
simultaneously:  a language like \TeX\ for formatting, and a language like
\Cee\ for programming. Neither type of language can provide the
best documentation by itself. But when both are appropriately combined, we
obtain a system that is much more useful than either language separately.

The structure of a software program may be thought of as a ``web'' that is
made up of many interconnected pieces. To document such a program, we want
to explain each individual part of the web and how it relates to its
neighbors. The typographic tools provided by \TeX\ give us an opportunity
to explain the local structure of each part by making that structure
visible, and the programming tools provided by \Cee\ make it possible
for us to specify the algorithms formally and unambiguously. By combining
the two, we can develop a style of programming that maximizes our ability
to perceive the structure of a complex piece of software, and at the same
time the documented programs can be mechanically translated into a working
software system that matches the documentation.

The \.{WEB} system consists of two programs named \.{WEAVE} and \.{TANGLE}.
More precisely, \.{CWEB} calls them \.{CWEAVE} and \.{CTANGLE}.
When writing a \.{CWEB} program the user keeps the
\Cee\ code and the documentation in the same file, called the \.{WEB}
file and generally named \.{something.w}.  The command
`\.{cweave something}' creates an output file \.{something.tex}, which
can then be fed to \TeX, yielding a ``pretty printed'' version of
\.{something.w} that correctly handles
typographic details like page layout and the use of indentation,
italics, boldface, and mathematical symbols. The typeset output also
includes extensive cross-index
information that is gathered automatically.  Similarly, if you run the
command `\.{ctangle something}' you will get a \Cee\ file \.{something.c},
with can then be compiled to yield executable code.

Besides providing a documentation tool, \.{WEB} enhances the \Cee\
language by providing the
ability to permute pieces of the program text, so that a large system can
be understood entirely in terms of small sections and their local
interrelationships.  The \.{TANGLE} program is so named because it takes a
given web and moves the sections from their web structure into the order
required by \Cee; the advantage of programming in \.{WEB} is that the
algorithms can be expressed in ``untangled'' form, with each section
explained separately.  The \.{WEAVE} program is so named because it takes
a given web and intertwines the \TeX\ and \Cee\ portions contained in
each section, then it knits the whole fabric into a structured document.
(Get it? Wow.)  Perhaps there is some deep connection here with the fact
that the German word for ``weave'' is ``{\it web\/}'', and the
corresponding Latin imperative is ``{\it texe\/}''!

A user of \.{CWEB} should be fairly familiar with the \Cee\
programming language.  A minimal amount of acquaintance with \TeX\ is also
desirable, but in fact it can be acquired as one uses \.{WEB}, since
straight text can be typeset in \TeX\ with virtually no knowledge of
that language.  To someone familiar with both \Cee\ and \TeX\ the amount of
effort necessary to learn the commands of \.{WEB} is small.

\section The language.
Two kinds of material go into \.{WEB} files: \TeX\ text and \Cee\ text.
A programmer writing in \.{WEB} should be thinking both of the
documentation and of the \Cee\ program being created;
i.e., the programmer should be instinctively aware of the different
actions that \.{WEAVE} and \.{TANGLE} will perform on the \.{WEB} file.
\TeX\ text is essentially copied without change by \.{WEAVE}, and it is
entirely deleted by \.{TANGLE}; the \TeX\ text is ``pure
documentation.'' \Cee\ text, on the other hand, is formatted by
\.{WEAVE} and it is shuffled around by \.{TANGLE}, according to rules that
will become clear later. For now the important point to keep in mind is
that there are two kinds of text. Writing \.{WEB} programs is something
like writing \TeX\ documents, but with an additional ``\Cee\ mode''
that is added to \TeX's horizontal mode, vertical mode, and math mode.

A \.{WEB} file is built up from units called {\sl sections\/} that are more
or less self-contained.  Each section has three parts:

\yskip\item{1)} A \TeX\ part, containing explanatory material about what
is going on in the section.

\item{2)} A middle part, containing macro definitions that serve as
abbreviations for \Cee\ constructions that would be less comprehensible
if written out in full each time. They are turned by \.{TANGLE} into
preprocessor macro definitions.

\item{3)} A \Cee\ part, containing a piece of the program that
\.{TANGLE} will produce. This \Cee\ code should ideally be about a
dozen lines long, so that it is easily comprehensible as a unit and so
that its structure is readily perceived.

\yskip\noindent The three parts of each section must appear in this order;
i.e., the \TeX\ commentary must come first, then the middle part, and
finally the \Cee\ code. Any of the parts may be empty.

A section begins with either of the symbols `\.{@\ }' or `\.{@*}', where
`\.{\ }' denotes a blank space. A section ends
at the beginning of the next section (i.e., at the next
`\.{@\ }' or `\.{@*}'), or at the end of the file, whichever comes first.
The \.{WEB} file may also contain material that is not part of any section
at all, namely the text (if any) that occurs before the first section.
Such text is said to be ``in limbo''; it is ignored by \.{TANGLE}
and copied essentially verbatim by \.{WEAVE}, so its function is to
provide any additional formatting instructions that may be desired in the
\TeX\ output. Indeed, it is customary to begin a \.{WEB} file with
\TeX\ code in limbo that loads special fonts, defines special macros,
changes the page sizes, and/or produces a title page.

Sections are numbered consecutively, starting with 1. These numbers appear
at the beginning of each section of the \TeX\ documentation output by
\.{WEAVE}, and they appear
as bracketed comments at the beginning and end of the code generated by that
section in the \Cee\ program output by \.{TANGLE}.

Fortunately, you never mention these numbers yourself when you are writing
in \.{WEB}. You just say `\.{@\ }' or `\.{@*}' at the beginning of each
new section, and the numbers are supplied automatically by \.{WEAVE} and
\.{TANGLE}. As far as you are concerned, a section has a
{\sl name\/} instead of a number; its name is specified by writing
`\.{@<}' followed by \TeX\ text followed by `\.{@>}'. When \.{WEAVE}
outputs a section name, it replaces the `\.{@<}' and `\.{@>}' by
angle brackets and inserts the section number in small type. Thus, when you
read the output of \.{WEAVE} it is easy to locate any section that is
referred to in another section.

For expository purposes, a section name should be a good description of the
contents of that section; i.e., it should stand for the abstraction
represented by the section. Then the section can be ``plugged into'' one or
more other sections in such a way
that unimportant details of its inner workings
are suppressed.  A section name therefore ought to be long enough to convey
the necessary meaning.

Unfortunately, it is laborious to type
such long names over and over again, and it is also difficult to specify a
long name twice in exactly the same way so that \.{WEAVE} and \.{TANGLE}
will be able to match the names to the sections. To ameliorate this difficulty,
\.{WEAVE} and \.{TANGLE} let you abbreviate a section name, so long as
the full name appears somewhere in the \.{WEB} file; you can type simply
`\.{@<$\alpha$...@>}', where $\alpha$ is any string that is a prefix of
exactly one section name appearing in the file. For example, `\.{@<Clear
the arrays@>}' can be abbreviated to `\.{@<Clear...@>}' if no other section
name begins with the five letters `\.{Clear}'.  Elsewhere
you might use the abbreviation `\.{@<Clear t...@>}', and so on.

Section names must otherwise match character for character, except
that consecutive characters of white space (spaces, tab marks, newlines, and/or
form feeds) are treated as equivalent to a single space, and such spaces are
deleted at the beginning and end of the name. Thus, `\.{@< Clear { }the
arrays @>}' will also match the name in the previous example.
Spaces following the ellipsis in abbeviations are ignored as well, but
not those before, so that `\.{@<Clear t ...@>}' would not match
`\.{@<Clear the arrays@>}'.

We have said that a section begins with `\.{@\ }' or `\.{@*}', but we
didn't say how it gets divided up into a \TeX\ part, a middle part,
and a \Cee\ part. The middle part begins with the first appearance of
`\.{@d}' or `\.{@f}' in the section, and the \Cee\ part begins with
the first appearance of `\.{@c}' or `\.{@<section name@>=}'.  In the
latter case you are saying, in effect, that the section name stands
for the \Cee\ text that follows. Alternatively, if the \Cee\ part
begins with `\.{@c}' instead of a section name, the current section is
said to be {\sl unnamed}.

The construct `\.{@<section name@>}' can appear
any number of times in the \Cee\ part of a section:
subsequent appearences indicate that the named section is being
``used'' rather than ``defined'', that is, that the
\Cee\ code for the named section, presumably defined elsewhere, should be
spliced in at this point in the \Cee\ program.  Indeed,
the main idea of \.{TANGLE} is to make a \Cee\ program out of
individual sections, named and unnamed.  The exact way in which this is done
is the following: First all the macro definitions
indicated by `\.{@d}' are turned into \Cee\ preprocessor macro definitions
and copied at the beginning.
Then the \Cee\ parts of unnamed sections are copied down,
in order; this constitutes the initial
approximation $T_0$ to the text of the program. (There should be at least
one unnamed section, otherwise there will be no program.) Then all section
names that appear in the initial text $T_0$ are replaced by the \Cee\
parts of the corresponding sections, and this substitution process
continues until no section names remain. All comments are removed, because
the \Cee\ program is intended only for the eyes of the \Cee\ compiler.

If the same name has been given to more than one section, the \Cee\ text
for that name is obtained by putting together all of the \Cee\ parts in
the corresponding sections. This feature is useful, for example, in a
section named `Global variables', since one can then
declare global variables in whatever sections those variables are
introduced. When several sections have the same name, \.{WEAVE} assigns the
first section number as the number corresponding to that name, and it
inserts a note at the bottom of that section telling the reader to `See
also sections so-and-so'; this footnote gives the numbers of all the other
sections having the same name as the present one. The \Cee\ text
corresponding to a section is usually formatted by \.{WEAVE} so that the
output has an equivalence sign in place of the equals sign in the \.{WEB}
file; i.e., the output says `$\langle\,$section
name$\,\rangle\equiv\null$\Cee\ text'. However, in the case of the second
and subsequent appearances of a section with the same name, this `$\equiv$'
sign is replaced by `$\mathrel+\equiv$', as an indication that the
following \Cee\ text is being appended to the \Cee\ text of another section.

As \.{TANGLE} enters and leaves sections, it inserts preprocessor
\.{\#line} commands into the \Cee\ output file.  This means that
when the compiler gives you error messages, or when you debug your program,
the messages refer to line numbers in the \.{WEB} file, and not in the
\Cee\ file.  In most cases you can therefore
forget about the \Cee\ file altogether.

The general idea of \.{WEAVE} is to make a \.{.tex} file from the \.{WEB}
file in the following way: The first line of the \.{.tex} file
tells \TeX\ to input a file with macros that
define \.{CWEB}'s documentation conventions. The next lines of the file
will be copied from whatever \TeX\ text is in limbo before the first
section.  Then comes the output for each section in turn, possibly
interspersed with end-of-page marks.  Finally, \.{WEAVE} will generate a
cross-reference index that lists each section number in which each \Cee\
identifier appears, and it will also generate an alphabetized list
of the section names, as well as a table of contents that
shows the page and section numbers for each ``starred'' section.

What is a ``starred'' section, you ask? A section that begins with `\.{@*}'
instead of `\.{@\ }' is slightly special in that it denotes a new major
group of sections. The `\.{@*}' should be followed by the title of this
group, followed by a period. Such sections will always start on a new page
in the \TeX\ output, and the group title will appear as a running headline
on all subsequent pages until the next starred section. The title will also
appear in the table of contents, and in boldface type at the beginning of
its section. Caution:  Do not use \TeX\ control sequences in such titles,
unless you know that the \.{cwebmac} macros will do the right thing with
them. The reason is that these titles are converted to uppercase when
they appear as running heads, and they are converted to boldface when they
appear at the beginning of their sections, and they are also written out to
a table-of-contents file used for temporary storage while \TeX\ is
working; whatever control sequences you use must be meaningful in all
three of these modes.

The \TeX\ output produced by \.{WEAVE} for each section consists of
the following: First comes the section number (e.g., `\.{\\M123.}'
at the beginning of section 123, except that `\.{\\N}' appears in place of
`\.{\\M}' at the beginning of a starred section). Then comes the
\TeX\ part of the section, copied almost verbatim except as noted
below. Then comes the middle part and the \Cee\ part, formatted
so that there will be a little extra space between them if both are
nonempty. The middle and \Cee\ parts are obtained by inserting
a bunch of funny-looking \TeX\ macros into the \Cee\ program; these
macros handle typographic details about fonts and proper math spacing,
as well as line breaks and indentation.

When you are typing \TeX\ text, you will probably want to make frequent
reference to variables and other quantities in your \Cee\ code, and you
will want those variables to have the same typographic treatment
when they appear in your text as when they appear in your
program.  Therefore the \.{WEB} language allows you to get the effect of
\Cee\ editing within \TeX\ text, if you place `\.|' marks before and
after the \Cee\ material. For example, suppose you want to say something
like this:
$$\hbox{ If \\{pa} is declared as `\&{int} ${}{*}\\{pa}$',
the assignment $\\{pa}\K{\AND}\|a[\T{0}]$ makes \\{pa}
point to the zeroth element of \|a.}$$
The \TeX\ text would look like this in your \.{WEB} file:
$$\lpile{\.{If |pa| is declared as `|int *pa|', the}\cr
\.{assignment |pa=\&a[0]| makes |pa| point
to the zeroth element of |a|.}\cr}$$
And \.{WEAVE} translates this into something you are glad you didn't have
to type:
$$\lpile{\.{If \\\\\{pa\} is declared as
  `\\\&\{int\} \$\{\}\{*\}\\\\\{pa\}\$',}\cr
\.{the assignment \$\\\\\{pa\}\\K\{\\AND\}\\|a[\\T\{0\}]\$}\cr
\.{makes \\\\\{pa\} point to the zeroth element of \\|a.}\cr}$$
Incidentally, the cross-reference index that \.{WEAVE} would make, in
the presence of a comment like this, would include
the current section number as one of the index entries for \\{pa},
even though \\{pa} might not appear in the \Cee\ part of
this section. Thus, the index covers references to identifiers in
the explanatory comments as well as in the program itself; you will
soon learn to appreciate this feature. However, the identifiers
\&{int} and \|a\ would not be indexed,
because \.{WEAVE} does not make index entries for reserved words or
single-letter identifiers. Such identifiers are felt to be so ubiquitous
that it would be pointless to mention every place where they occur.

Although a section begins with \TeX\ text and ends with \Cee\ text, we
have noted that the dividing line isn't sharp, since \Cee\ text can be
included in \TeX\ text if it is enclosed in `\pb'.  Conversely, \TeX\ text
also appears frequently within \Cee\ text, because everything in
comments (i.e., between \.{/*} and \.{*/}) is treated as \TeX\ text.
Furthermore, a section name, like a comment, is expected to be found
in \Cee\ text, but the name itself consists of \TeX\ text; thus, a \.{WEB} file
typically involves constructions like `\.{if} \.{(x==0)}
\.{@<Empty} \.{the} \.{|buffer|} \.{array@>}' where we go back and forth
between \Cee\ and \TeX\ conventions in a natural way.
To include a section name in \TeX\ text, that is, in the \TeX\ part of
a section or in a \Cee\ comment, you can enclose it in `\pb'.  In this case
the section name is being ``cited'', rather than defined or used.

\section Macros.
The control code \.{@d} followed by
$$\\{identifier}\.{ }\hbox{\Cee\ text}\qquad\hbox{or by}\qquad
\\{identifier}\.(\\{par}_1,\ldots,\\{par}_n\.{) }\hbox{\Cee\ text}$$
(where there is no blank between the
\\{identifier} and the parentheses in the second case) is
transformed by \.{TANGLE} into a preprocessor command, starting with
\.{\#define}, which is printed at the top of the \Cee\ output file
as explained earlier.

A `\.{@d}' macro definition can go on for several lines, and the
newlines don't have to be protected by backslashes, since \.{TANGLE}
itself inserts the backslashes.   If
for any reason you need a \.{\#define} command at a specific spot in
your \Cee\ file, you can treat it as \Cee\ code, instead of as a
\.{WEB} macro; but then you do have to protect newlines yourself.

\section Strings and constants.
If you want a string to appear in the \Cee\ file, delimited by pairs of
\.' or \." marks as usual, you can type it exactly so in the \.{WEB} file,
except that the character `\.@' should be typed `\.{@@}' (it becomes a
control code, the only one that can appear in strings; see below).
Strings should end on the same line as they begin, unless there's a
backslash at the end of lines within them.

\TeX\ and \Cee\ have different ways to refer to octal and hex constants,
because \TeX\ is oriented to technical writing while \Cee\ is oriented to
computer processing.  In \TeX\ you
make a constant octal or hexadecimal by prepending \.' or \.",
respectively, to it; in \Cee\ the constant should be preceded by \.0
or \.{0x}.  In \.{WEB} it seems reasonable to let each convention hold
in its respective realm; so in \Cee\ text you get $40_8$ by typing
`\.{040}', which \.{TANGLE} faithfully copies into the \Cee\ file (for
the compiler's benefit) and which \.{WEAVE} prints as $\T{\~40}$.
Similarly, \.{WEAVE} prints the hexadecimal \Cee\ constant `\.{0x20}'
as \T{\^20}. The use of italic font for octal digits and typewriter font
for hexadecimal digits makes the meaning of such constants clearer in
a document. For consistency, then, you
should type `\.{|040|}'  or `\.{|0x20|}'
in the \TeX\ part of the section.

\section Control codes.
A \.{WEB} {\sl control code\/}
is a two-character combination of which the first is `\.@'.
We've already seen the meaning of several control codes; here is a
complete list of all of them.

The letters $L$, $T$,
$C$, $\\{Se}$, $\\{Co}$, and/or $S$ following each code indicate whether or not that
code is allowable in limbo, in \TeX\ text, in \Cee\ text, in section
names, in comments, and/or in strings.  A bar over such a letter means
that the control code terminates the present part of the \.{WEB} file; for
example, $\overline L$ means that this control code ends the limbo material
before the first section.

\gdef\@#1[#2] {\penalty-100\yskip\hangindent 2em\noindent\.{@#1\unskip
  \spacefactor1000{ }}$[#2]$\quad}
\def\more{\hangindent 2em \hangafter0}
\def\oP{\overline C}
\def\oT{\overline T\mskip1mu}

\@@ [\\{Co},L,\\{Se},C,S,T] A double \.@ denotes the single character `\.@'. This is
the only control code that is legal in limbo, in comments, and in strings.
Note that you must use this convention if you are giving an internet
email address in a \.{WEB} file (e.g., \.{levy@@math.berkeley.edu}).

\@\ [\overline L,\oP,\oT] This denotes the beginning of a new
(unstarred) section. A tab mark or form feed or end-of-line character
is equivalent to a space when it follows an \.@ sign (and in most
other cases).

\@* [\overline L,\oP,\oT] This denotes the beginning of a new starred
section, i.e., a section that begins a new major group. The title of the new
group should appear after the \.{@*}, followed by a period. As explained
above, \TeX\ control sequences should be avoided in such titles unless
they are quite simple. When \.{WEAVE} and \.{TANGLE} read a \.{@*}, they
print an asterisk on the terminal
followed by the current section number, so that the user
can see some indication of progress. The very first section should be starred.

% This should be outdented, since it applies to more several items.
%
\penalty-100\yskip\noindent
The middle part of each section consists of any number of
macro definitions (beginning with \.{@d}) and format definitions (beginning
with \.{@f} or \.{@s}), intermixed in any order.

\@d [\oP,\oT] Macro definitions begin with \.{@d} (or \.{@D}), followed by
an identifier and optional parameters and \Cee\ text as explained earlier.

\@f [\oP,\oT] Format definitions begin with \.{@f} (or \.{@F}); they cause
\.{WEAVE} to treat identifiers in a special way when they appear in
\Cee\ text. The general form of a format definition is `\.{@f} \|l
\|r', followed by an optional comment enclosed between
\.{/*} and \.{*/}, where \|l and \|r
are identifiers; \.{WEAVE} will subsequently treat identifier \|l as it
currently treats \|r. This feature allows a \.{WEB} programmer to invent
new reserved words and/or to unreserve some of \Cee's reserved
identifiers. If \|r is the special identifier `\\{TeX}', identifier \|l
will be formatted as a \TeX\ control sequence; for example,
`\.{@f foo TeX}' in the \.{WEB} file will cause identifier \\{foo} to
be output as \.{\\foo} by \.{WEAVE}. The programmer should define
\.{\\foo} to have whatever custom format is desired, assuming \TeX\
math mode. (Each underline
character is converted to \.{x} when making the \TeX\ control sequence;
thus \\{foo\_bar} becomes \.{\\fooxbar}.)

\more \.{WEAVE} knows that identifiers being
defined with a \&{typedef} should become reserved words; thus you
don't need format definitions very often.

\@s [\oP,\oT] Same as \.{@f}, but \.{WEAVE} does not show the format
definition in the output. This is used mostly in \.{@i} files.

\@c [\oP,\oT] The \Cee\ part of an unnamed section begins with \.{@c}
(or \.{@C}). This causes \.{TANGLE} to append the following \Cee\ code
to the initial program text $T_0$ as explained above. The \.{WEAVE}
processor does not cause a `\.{@c}' to appear explicitly in the \TeX\
output, so if you are creating a \.{WEB} file based on a \TeX-printed
\.{WEB} documentation you have to remember to insert \.{@c} in the
appropriate places of the unnamed sections.

\@p [\oP,\oT] The \Cee\ part of an unnamed section may alternatively begin
with \.{@p} (or \.{@P}), for compatibility with other \.{WEB} systems.
\.{CWEB} treats \.{@c} and \.{@p} identically.

\more Because of the rules by which every section is broken into three parts,
the control codes `\.{@d}', `\.{@f}', `\.{@c}', and `\.{@p}'
are not allowed to occur once the \Cee\ part of a section has begun.

\@< [C,\oT] A section name (or unambiguous prefix, as discussed above)
is delimited by \.{@<} and \.{@>}, and consists of \TeX\ text;
but the whole construct \.{@<...@>} is conceptually a \Cee\ element.
(In this sense a section name is like a \Cee\ comment.)
The \TeX\ text should not contain any \.{WEB} control codes
except \.{@@}, unless these control codes appear in \Cee\ text that
is delimited by \pb.  An occurrence of a section name may indicate
that the section is being defined, used, or cited.

\more
A \.{@<} appearing in the \TeX\ part (or the middle part) of a section
inaugurates the \Cee\ part of the section, and defines the following section
name to stand for the subsequent \Cee\ code. The closing \.{@>} should be
followed by \.{=} or \.{+=}.

\more
In the \Cee\ part of a section, \.{@<...@>} indicates that the named
section is being used (its \Cee\ definition is spliced in by
\.{TANGLE}, as explained above).  As an error-detection measure,
\.{TANGLE} and \.{WEAVE} complain if such a section name is followed
by \.=, because most likely this is meant as the definition of a new
section, and so should be preceded by \.{@\ }.  If you really want to
say $\langle\,$foo$\,\rangle=\\{bar}$, where $\langle\,$foo$\,\rangle$
is being used and not defined, put a newline before the \.=.

\more
Finally, \.{|@<...@>|} appearing in a \TeX\ context (in the \TeX\ part
of a section, or in a comment) means that the named section is being
referred to.  Such an occurrence is ignored by \.{TANGLE}. Note that
even here we think of the section name as being a \Cee\ element, hence
the \pb.  There is one \TeX\ context where
\.{|@<...@>|} is not allowed: inside another section name.

\@( [C,\oT] A section name can begin with \.{@(} instead of \.{@<}.
Everything works exactly as before, except that \.{@(foo@>} denotes
a special section name all of whose \Cee\ code is written by \.{TANGLE}
to file \.{foo}. In this way you can get multiple-file output from
a single \.{WEB} file. (The \.{@d} definitions are not output
to such files, only to the master \.{.c} file.)

\@' [C] This control code is dangerous because it has quite different
meanings in \.{CWEB} and the original \.{WEB}. In \.{CWEB} it produces the
decimal constant corresponding to the ASCII code for a string of length~1
(e.g., \.{@'a'} is \.{TANGLE}d into \.{97} and \.{@'\\t'} into
\.9). You might want to use this if you need to work in ASCII on a
non-ASCII machine; but in most cases the \Cee\ conventions of
\.{<ctype.h>} are adequate for character-set-independent programming.

\@\& [C] The \.{@\&} operation causes whatever is on its left to be
adjacent to whatever is on its right, in the \Cee\ output. No spaces or
line breaks will separate these two items.

\penalty-100\yskip \noindent
The next several control codes introduce ``control
texts'', which end with the next `\.{@>}'.  The closing `\.{@>}' must
be on the same line of the \.{WEB} file where the control text began.
Furthermore, no \.{WEB} control codes are allowed in a control text,
not even \.{@@}. (If you need an \.{@} sign you can get around this
restriction by typing `\.{\{\\AT\}}'.)

\@\^ [C,T] The control text that follows, up to the next
`\.{@>}', will be entered into the index together with the identifiers of
the \Cee\ program; this text will appear in roman type. For example, to
put the phrase ``system dependencies'' into the index, type
`\.{@\^system dependencies@>}' in each section
that you want to index as system dependent.

\@. [C,T] The control text that follows will be entered into the index
in \.{typewriter} \.{type}.

\@: [C,T] The control text that follows will be entered into the index
in a format controlled by the \TeX\ macro `\.{\\9}', which you
should define as desired.

\@t [C] The control text that follows will
be put into a \TeX\ \.{\\hbox} and formatted along with the neighboring
\Cee\ program. This text is ignored by \.{TANGLE}, but it can be used
for various purposes within \.{WEAVE}. For example, you can make comments
that mix \Cee\ and classical mathematics, as in `$\\{size}<2^{15}$', by
typing `\.{|size < @t\$2\^\{15\}\$@>|}'.

\@= [C] The control text that follows will
be passed verbatim to the \Cee\ program.

\@q [C,T] The control text that follows will
be totally ignored; it's a comment for readers of the \.{WEB} file only.

\@! [C,T] The section number in an index entry will be underlined if `\.{@!}'
immediately precedes the identifier or control text being indexed. This
convention is used to distinguish the sections where an identifier is
defined, or where it is explained in some special way, from the sections
where it is used. A~reserved word or an identifier of length one will not
be indexed except for underlined entries. An `\.{@!}' is implicitly inserted
by \.{WEAVE} when an identifier is being defined or declared in \Cee\
code; for example, the definition
$$\hbox{\&{int} \\{array}[\\{max\_dim}], \\{count}${}=\\{old\_count};$}$$
makes the names \\{array} and \\{count} get an underlined entry in the
index.  Statement labels, function definitions like
$\\{main}(\\{argc},\39\\{argv})$, and \&{typedef} definitions also
imply underlining. A function definition doesn't define its arguments;
the arguments will, however, be defined
(i.e., their index entries will be underlined), if their types are
declared before the body of the function in the usual way
(e.g., `\.{int}~\\{argc}; \.{char}~${**}\\{argv}$; $\{\,\ldots\,\}$').

\yskip\noindent
The next eight control codes (namely `\.{@,}', `\.{@/}', `\.{@|}', `\.{@\#}',
`\.{@+}', `\.{@;}', `\.{@[}', and `\.{@]}') have no effect on the \Cee\
program output by \.{TANGLE}; they merely help to improve the readability
of the \TeX-formatted \Cee\ that is output by \.{WEAVE}, in unusual
circumstances. \.{WEAVE}'s built-in formatting method is fairly good
when dealing with syntactically correct \Cee\ text, but
it is incapable of handling all possible cases, because it must deal with
fragments of text involving macros and section names; these fragments do
not necessarily obey \Cee's syntax. Although \.{WEB} allows you to
override the automatic formatting, your best strategy is not to worry
about such things until you have seen what \.{WEAVE} produces automatically,
since you will probably need to make only a few corrections when you are
touching up your documentation.

\@, [C] This control code inserts a thin space in \.{WEAVE}'s output; it is
ignored by \.{TANGLE}. Sometimes you need this extra space if you are using
macros in an unusual way, e.g., if two identifiers are adjacent.

\@/ [C] This control code causes a line break to occur within a \Cee\
program formatted by \.{WEAVE}; it is ignored by \.{TANGLE}. Line breaks
are chosen automatically by \TeX\ according to a scheme that works 99\%\
of the time, but sometimes you will prefer to force a line break so that
the program is segmented according to logical rather than visual
criteria.

\@| [C] This control code specifies an optional line break in the midst of
an expression. For example, if you have
a long expression on the right-hand side of an assignment
statement, you can use `\.{@|}' to specify breakpoints more logical than
the ones that \TeX\ might choose on visual grounds.

\@\# [C] This control code forces a line break, like \.{@/} does,
and it also causes a little extra white space to appear between the lines at
this break. You might use it, for example,
between groups of macro definitions that are logically separate but within
the same section. \.{CWEB} automatically inserts this extra space
between functions, between external declarations and functions, and
between declarations and statements within a function.

\@+ [C] This control code cancels a line break that might otherwise be
inserted by \.{WEAVE}, e.g., before the word `\&{else}', if you want to
put a short if--else construction on a single line. It is ignored by
\.{TANGLE}. If you say `\.{\{@+}' at the beginning of a compound statement
that is the body of a function, the first declaration or
statement of the function will appear on the same line as the
left brace, and it will be indented by the same amount as the
second declaration or statement on the next line.

\@; [C] This control code is treated like a semicolon, for formatting
purposes, except that it is invisible. You can use it, for example, after
a section name or macro when the \Cee\ text represented by that section or macro
is a compound statement or ends
with a semicolon. Consider constructions like
$$\lpile{\.{if (condition) macro @;}\cr
\.{else break;}\cr}$$
where \\{macro} is defined to be a compound statement (enclosed in braces).
This is a well-known infelicity of \Cee\ syntax.

\@{[} [C] Place this before a macro argument that isn't otherwise formatted
correctly.

\@] [C] Place this after a macro argument that isn't otherwise formatted
correctly.

\yskip\noindent
The next two entries describe ``meta'' control codes: they govern the
input that \.{WEB} sees.

\@{x @y @z}[\\{change\_file}]
\.{WEAVE} and \.{TANGLE} are designed to work with two input files,
called \\{web\_file} and \\{change\_file}, where \\{change\_file} contains
data that overrides selected portions of \\{web\_file}. The resulting merged
text is actually what has been called the \.{WEB} file elsewhere in this
report.

\more Here's how it works: The change file consists of zero or more ``changes,''
where a change has the form `\.{@x}$\langle$old lines$\rangle$\.{@y}$\langle$%
new lines$\rangle$\.{@z}'. The special control codes \.{@x}, \.{@y}, \.{@z},
which are allowed only in change files, must appear at the beginning of a line;
the remainder of such a line is ignored.
The $\langle$old lines$\rangle$ represent material that exactly matches
consecutive lines of the \\{web\_file}; the $\langle$new lines$\rangle$
represent zero or more lines that are supposed to replace the old. Whenever
the first ``old line'' of a change is found to match a line in the
\\{web\_file}, all the other lines in that change must match too.

\more Between changes, before the first change, and after the last change,
the change file can have any number of lines that do not begin with
`\.{@x}', `\.{@y}', or~`\.{@z}'. Such lines are bypassed and not used for
matching purposes.

\more This dual-input feature is useful when working with a master \.{WEB} file
that has been received from elsewhere (e.g., \.{tangle.w} or
\.{weave.w} or \.{tex.web}), when changes are desirable to customize the
program for your local computer system. You will be able to debug your
system-dependent changes without clobbering the master web file; and once
your changes are working, you will be able to incorporate them readily
into new releases of the master web file that you might receive from time
to time.

\@i [\\{web\_file}]
Furthermore the \\{web\_file} itself can be a combination of
several files.  When either \.{CWEAVE} or \.{CTANGLE} is reading a file and
encounters the control code \.{@i} at the beginning of a line, it
interrupts normal reading and start looking at the file named after the
\.{@i}, much as the \Cee\ preprocessor does when it encounters an \.{\#include}
line.  After the included file has been entirely read, the program
goes back to the next line
of the original file.  The file name following \.{@i} can be
surrounded by \." characters, but such delimiters are
optional. Include files can nest.

\more
Change files can have lines starting with \.{@i}. In this way
you can replace one included file with another.  It is not possible
to replace individual lines of an included file using `\.{@x}', etc.;
conceptually, the
replacement mechanism described above does its work first, and
its output is then checked for \.{@i} lines.

\section Additional features and caveats.

1. In certain installations of \.{CWEB} that
{\def\\#1#2{`{\tentex\char'#1#2}'}%
have an extended character set, the characters
\\13, \\01, \\31, \\32, \\34, \\35,
\\36, \\37, \\04, \\20, and \\21}
can be typed as abbreviations for `\.{++}', `\.{--}', `\.{->}',
`\.{!=}', `\.{<=}', `\.{>=}', `\.{==}', `\.{\v\v}', `\.{\&\&}',
`\.{<<}', and `\.{>>}',
respectively.

2. If you have an extended character set, all of the characters listed
in Appendix~C of {\sl The \TeX book\/} can be used in strings. But you should
stick to standard ASCII characters if you want to write programs that will
be useful to all the poor souls out there who don't have extended
character sets.

3. The \TeX\ file output by \.{WEAVE} is broken into lines having at most
80 characters each. The algorithm that does this line breaking is unaware
of \TeX's convention about comments following `\.\%' signs on a line. When
\TeX\ text is being copied, the existing line breaks are copied as well,
so there is no problem with `\.\%' signs unless the original \.{WEB} file
contains a line more than eighty characters long or a line with \Cee\
text in \pb\ that expands to more than eighty characters long. Such lines
should not have `\.\%' signs.

4. \Cee\ text is translated by a ``bottom up'' procedure that
identifies each token as a ``part of speech'' and combines parts of speech
into larger and larger phrases as much as possible according to a special
grammar that is explained in the documentation of \.{WEAVE}. It is easy to
learn the translation scheme for simple constructions like single
identifiers and short expressions, just by looking at a few examples of
what \.{WEAVE} does, but the general mechanism is somewhat complex because
it must handle much more than \Cee\ itself. Furthermore the output
contains embedded codes that cause \TeX\ to indent and break lines as
necessary, depending on the fonts used and the desired page width. For
best results it is wise to avoid enclosing long \Cee\ texts in \pb, since the
indentation and line breaking codes are omitted when the \pb\ text is
translated from \Cee\ to \TeX. Stick to simple expressions or
statements.  If a \Cee\ preprocessor command is enclosed in \pb,
the \.\# that introduces it must be at the beginning of a line,
or \.{WEAVE} won't print it correctly.

5. Comments are not permitted in \pb\ text. After a `\.|'
signals the change from \TeX\ text to \Cee\ text, the next `\.|' that is
not part of a string or control text or section name ends the \Cee\ text.

6. A comment must have properly nested occurrences of left and right
braces, otherwise \.{WEAVE} will complain. But it
does try to balance the braces, so that \TeX\ won't foul up too much.

7. When you're debugging a program and decide to omit some of your
\Cee\ code, do NOT simply ``comment it out.'' Such comments are not
in the spirit of \.{WEB} documentation; they will appear to readers
as if they were explanations of the uncommented-out instructions.
Furthermore, comments of a program must be valid \TeX\ text; hence
\.{WEAVE} will get confused if you enclose \Cee\ statements in
\.{/*...*/} instead of in \.{/*|...|*/}.  If you must comment out
\Cee\ code, you can surround it with preprocessor commands
like \.{\#if 0==1} and \.{\#endif}.

8. The \.{@f} feature allows you to define one identifier to act like
another, and these format definitions are carried out sequentially.
In general, a given identifier has only one printed format
throughout the entire document, and this format is used even before
the \.{@f} that defines it. The reason is that \.{WEAVE} operates in two
passes; it processes \.{@f}'s and cross-references on the first pass and
it does the output on the second.  (However, identifiers that
implicitly get a boldface format, thanks to a \.{typedef} declaration,
don't obey this rule: they are printed differently before and after
the relevant \.{typedef}.  This is unfortunate, but hard to fix. You can
get around the restriction by using \.{@s}, before or after the \.{typedef}.)

9. Sometimes it is desirable to insert spacing into \Cee\ code that is
more general than the thin space provided by `\.{@,}'. The \.{@t} feature
can be used for this purpose; e.g., `\.{@t\\hskip 1in@>}' will
leave one inch of blank space. Furthermore, `\.{@t\\4@>}' can be
used to backspace by one unit of indentation, since the control sequence
\.{\\4} is defined in \.{cwebmac} to be such a backspace. (This
control sequence is used, for example, at the beginning of lines that
contain labeled statements, so that the label will stick out a little at
the left.) You can also use `\.{@t\}\\3\{-5@>}' to force a break
in the middle of an expression.

\section Running the programs.
The \UNIX\ command line for \.{CTANGLE} is
$$\.{ctangle [options] web\_file[.w] [change\_file[.ch] [out\_file]]}$$
and the same conventions apply to \.{CWEAVE}. If no change file is
specified, the change file is null. The extensions \.{.w} and \.{.ch}
are appended only if the given file names contain no dot. If the
web file defined in this way cannot be found, the extension \.{.web}
will be tried. For example, `\.{cweave} \.{cob}' will try to read
\.{cob.w}; failing that, it will try \.{cob.web} before giving up.
If no output file name is specified, the name of the \Cee\ file output by
\.{CTANGLE} is obtained by appending the extension \.{.c};
the name of the \TeX\ file output by \.{CWEAVE} gets the extension \.{.tex}.

Programmers who like terseness might choose to set up their
 operating shell so that `\.{wv}' expands to
`\.{cweave -bhp}'; this will suppress most terminal output from \.{CWEAVE}
except for error messages.

Options are introduced either by a \.- sign, to turn an option off,
or by a \.+ sign to turn one on. For example, `\.{-fb}' turns off
options \.f and \.b; `\.{+s}' turns on option \.s. Options can be
specified before the file names, after the file names, or both. The following
options are currently implemented:

\yskip
\def\option#1 {\textindent{\.#1}\hangindent2\parindent}

\option b Print a banner line at the beginning of execution. (On
by default.)

\option f Force line breaks after each \Cee\ statement formatted
by \.{WEAVE}. (On by default; \.{-f} saves paper but looks less \Cee-like
to some people.) (Has no effect on \.{TANGLE}.)

\option h Print a happy message at the conclusion of a successful
run. (On by default.)

\option p Give progress reports as the program runs. (On by default.)

\option s Show statistics about memory usage after the program
runs to completion. (Off by default.) This feature works only if
the programs have been compiled with the \.{-DSTAT} switch. If you
have large \.{WEB} files or sections, you may need to see
how close you come to exceeding the capacity of \.{TANGLE} and/or \.{WEAVE}.

\option x Include indexes and a table of contents in the \TeX\ file
output by \.{WEAVE}. (On by default.) (Has no effect on \.{TANGLE}.)

\section Further details about formatting.
You may not like the way \.{WEAVE} handles certain
situations. If you're desperate, you can customize \.{WEAVE}
by changing its grammar.  This means changing the source code,
a task that you might find amusing. A table of grammar rules
appears in the \.{WEAVE} source listing, and you can make a separate
copy of that table by copying the file \.{prod.w} found in the \.{cweb}
sources and saying `\.{cweave}~\.{-x}~\.{prod}'.

If you compile \.{WEAVE} with the \.{-DDEBUG} option on the command
line, you will be able to see exactly
how \.{WEAVE} is parsing your \Cee\ code by preceding
it with the line `\.{@ @c @2}'. (The control code `\.{@2}'
turns on a ``peeping'' mode, and `\.{@0}' turns it off.)
For example, if you run \.{WEAVE} on the file
\medskip
\begingroup
\verbatim
@ @c @2
main (argc,argv)
char **argv;
{ for (;argc>0;argc--) printf("%s\n",argv[argc-1]); }
!endgroup
\endgroup
\medskip\noindent
you get the following gibberish on your screen:
\medskip
\begingroup
\verbatim
[...]
4:*exp ( +exp+ )...
8:*exp +exp+ int...
5:*+exp+ int +unorbinop+...
[...]
45: +fn_decl+*+{+ -stmt- +}-
40:*+fn_decl+ -stmt-
37:*+function-
[...]
!endgroup
\endgroup
\medskip
The first line says that grammar rule 4 has just been applied, and \.{WEAVE}
currently has in its memory a sequence of chunks of \TeX\ code (called
``scraps'') that are respectively
of type \\{exp} (for expression), open-parenthesis,
\\{exp} again, close-parenthesis, and further scraps that haven't yet
been considered by the parser.  (The \.+ and \.- signs stipulate that
\TeX\ should be in or out of math mode at the scrap boundaries. The \.* shows
the parser's current position.)
Then rule 8 is applied, and
the sequence $(\,exp\,)$ becomes an \\{exp} and so on.  In the
end the whole \Cee\ text has become one big scrap of type \\{function}.

Sometimes things don't work as smoothly, and you get a bunch of
lines lumped together.  This means that \.{WEAVE} could not
digest something in your \Cee\ code.  For instance, suppose
`\.{@<Argument definitions@>}' had appeared instead of
`\.{char **argv;}' in the program above. Then \.{WEAVE} would have
been somewhat mystified, since it thinks that section names
are just \\{exp}s.  Thus it would tell \TeX\ to format
`\X2:Argument declarations\X' on the same line as
`$\\{main}(\\{argc},\39\\{argv}{}$)'.
In this case you should help \.{WEAVE} by putting `\.{@/}' after
`\.{main(argc,argv)}'.

\.{CWEAVE} automatically inserts a bit of extra space between declarations
and the first apparent statement of a block. One way to defeat this is
$$\vbox{\halign{#\hfil\cr
\.{int x;@+@t@>@;@/}\cr
\.{@<Other locals@>@;@\#}\cr}}$$
the `\.{@\#}' will put extra space after `$\langle\,$Other locals$\,\rangle$'.

\section Acknowledgments.
The authors wish to thank all who contributed suggestions and criticism to
the development of \.{CWEB}. We are especially grateful to Norman Ramsey,
Joachim Schnitter, Klaus Guntermann, and Nelson Beebe, who contributed
code.  Ramsey also has made
literate programming accessible to users of yet other languages by means of
his \.{SPIDER} system [see {\sl Communications of the ACM\/ \bf32} (1989),
1051--1055].

\section Appendices.
The basic ideas of \.{WEB} can be understood most easily by looking at
examples of ``real'' programs. Appendix~A shows the \.{CWEB} input that
generated sections 17--18 of the \.{common.w} file, which contains
routines common to \.{CWEAVE} and \.{CTANGLE}.
Appendix~B shows the corresponding \Cee\ code output by \.{CTANGLE}.
Appendix~C shows the
corresponding \TeX\ code output by \.{CWEAVE},
and Appendix~D shows how that output looks when printed out.
Appendix E is the file that sets \TeX\ up to accept
the output of \.{CWEAVE}, and Appendix~F discusses how to use some of those
macros to vary the output formats.

\vfil\eject

\def\runninghead{APPENDIX A --- {\tentt WEB} FILE FORMAT}
\section Appendix A.
The following is an excerpt of the file \.{common.w},
which contains routines shared by \.{CWEAVE} and \.{CTANGLE}.
Note that some of the lines are indented to show the program structure.
The indentation is ignored by \.{WEAVE} and \.{TANGLE}, but users find
that \.{WEB} files are quite readable if they have some such indentation.

The reader should first compare Appendix~A to Appendix~B; then the
same material should be compared to Appendices~C and~D.

\vskip 6pt
\begingroup \def\tt{\eighttt} \baselineskip9pt
\verbatim
@ Procedure |prime_the_change_buffer| sets |change_buffer| in
preparation for the next matching operation. Since blank lines in the change
file are not used for matching, we have
|(change_limit==change_buffer && !!changing)| if and only if
the change file is exhausted. This procedure is called only when
|changing| is 1; hence error messages will be reported correctly.

@<Func...@>=
prime_the_change_buffer()
{
  change_limit=change_buffer; /* this value is used if the change file ends */
  @<Skip over comment lines in the change file; |return| if end of file@>;
  @<Skip to the next nonblank line; |return| if end of file@>;
  @<Move |buffer| and |limit| to |change_buffer| and |change_limit|@>;
}

@ While looking for a line that begins with \.{@@x} in the change file, we
allow lines that begin with \.{@@}, as long as they don't begin with \.{@@y},
\.{@@z} or \.{@@i} (which would probably mean that the change file is fouled up).

@<Skip over comment lines in the change file...@>=
while(1) {
  change_line++;
  if (!!input_ln(change_file)) return;
  if (limit<buffer+2) continue;
  if (buffer[0]!!='@@') continue;
  if (isupper(buffer[1])) buffer[1]=tolower(buffer[1]);
  if (buffer[1]=='x') break;
  if (buffer[1]=='y' || buffer[1]=='z' || buffer[1]=='i') {
    loc=buffer+2;
    err_print("!! This should only appear after a @@x");
@.This should only appear...@>
  }
}

@ Here we are looking at lines following the \.{@@x}.

@<Skip to the next nonblank line...@>=
do {
  change_line++;
  if (!!input_ln(change_file)) {
    err_print("!! Change file ended after @@x");
@.Change file ended...@>
    return;
  }
} while (limit==buffer);

@ @<Move |buffer| and |limit| to |change_buffer| and |change_limit|@>=
{
  change_limit=change_buffer-buffer+limit;
  strncpy(change_buffer,buffer,limit-buffer+1);
}

!endgroup
\endgroup
\vfill\eject

\def\runninghead{APPENDIX B  --- TRANSLATION BY {\tentt TANGLE}}
\section Appendix B.
Here's the portion of the \Cee\ code generated by \.{TANGLE} that corresponds
to Appendix~A.  Notice that sections~13, 14 and~15
have been tangled into section~12.

\vskip6pt
\begingroup \def\tt{\eighttt} \baselineskip9pt
\verbatim
/*:9*//*12:*/
#line 240 "common.w"

prime_the_change_buffer()
{
change_limit= change_buffer;
/*13:*/
#line 253 "common.w"

while(1){
change_line++;
if(!!input_ln(change_file))return;
if(limit<buffer+2)continue;
if(buffer[0]!!='@')continue;
if(isupper(buffer[1]))buffer[1]= tolower(buffer[1]);
if(buffer[1]=='x')break;
if(buffer[1]=='y'||buffer[1]=='z'||buffer[1]=='i'){
loc= buffer+2;
err_print("!! This should only appear after a @x");

}
}

/*:13*/
#line 244 "common.w"
;
/*14:*/
#line 270 "common.w"

do{
change_line++;
if(!!input_ln(change_file)){
err_print("!! Change file ended after @x");

return;
}
}while(limit==buffer);

/*:14*/
#line 245 "common.w"
;
/*15:*/
#line 280 "common.w"

{
change_limit= change_buffer-buffer+limit;
strncpy(change_buffer,buffer,limit-buffer+1);
}

/*:15*/
#line 246 "common.w"
;
}

/*:12*//*16:*/
!endgroup
\endgroup
\vfill\eject

\def\runninghead{APPENDIX C --- TRANSLATION BY {\tentt WEAVE}}
\section Appendix C.
This excerpt from \.{common.tex} corresponds to Appendix A.

\vskip6pt
\begingroup \def\tt{\eighttt} \baselineskip9pt
\verbatim
\M12. Procedure \\{prime\_the\_change\_buffer} sets \\{change\_buffer} in
preparation for the next matching operation. Since blank lines in the change
file are not used for matching, we have
$(\\{change\_limit}\E\\{change\_buffer}\W\R\\{changing})$ if and only if
the change file is exhausted. This procedure is called only when
\\{changing} is 1; hence error messages will be reported correctly.

\Y\B\4\X4:Functions\X${}\mathrel+\E{}$\6
\\{prime\_the\_change\_buffer}(\,)\1\1\2\2\6
${}\{{}$\1\6
${}\\{change\_limit}\K\\{change\_buffer}{}$;\C{ this value is used if the
change file ends }\6
\X13:Skip over comment lines in the change file; \&{return} if end of file\X;\6
\X14:Skip to the next nonblank line; \&{return} if end of file\X;\6
\X15:Move \\{buffer} and \\{limit} to \\{change\_buffer} and \\{change\_limit}%
\X;\6
\4${}\}{}$\2\par \fi

\M13. While looking for a line that begins with \.{@x} in the change file, we
allow lines that begin with \.{@}, as long as they don't begin with \.{@y},
\.{@z} or \.{@i} (which would probably mean that the change file is fouled up).

\Y\B\4\X13:Skip over comment lines in the change file; \&{return} if end of
file\X${}\E{}$\6
\&{while} (\T{1})\5
${}\{{}$\1\6
${}\\{change\_line}\PP;{}$\6
\&{if} ${}(\R\\{input\_ln}(\\{change\_file})){}$\1\5
\&{return};\2\6
\&{if} ${}(\\{limit}<\\{buffer}+\T{2}){}$\1\5
\&{continue};\2\6
\&{if} ${}(\\{buffer}[\T{0}]\I\.{'@'}){}$\1\5
\&{continue};\2\6
\&{if} (\\{isupper}(\\{buffer}[\T{1}]))\1\5
${}\\{buffer}[\T{1}]\K\\{tolower}(\\{buffer}[\T{1}]);{}$\2\6
\&{if} ${}(\\{buffer}[\T{1}]\E\.{'x'}){}$\1\5
\&{break};\2\6
\&{if} ${}(\\{buffer}[\T{1}]\E\.{'y'}\V\\{buffer}[\T{1}]\E\.{'z'}\V\\{buffer}[%
\T{1}]\E\.{'i'}){}$\5
${}\{{}$\1\6
${}\\{loc}\K\\{buffer}+\T{2};{}$\6
\\{err\_print}(\.{"!!\ This\ should\ only\ }\)\.{appear\ after\ a\ @x"});\6
\4${}\}{}$\2\6
\4${}\}{}$\2\par
\U12.\fi

\M14. Here we are looking at lines following the \.{@x}.

\Y\B\4\X14:Skip to the next nonblank line; \&{return} if end of file\X${}\E{}$\6
\&{do}\6
${}\{{}$\1\6
${}\\{change\_line}\PP;{}$\6
\&{if} ${}(\R\\{input\_ln}(\\{change\_file})){}$\5
${}\{{}$\1\6
\\{err\_print}(\.{"!!\ Change\ file\ ended}\)\.{\ after\ @x"});\6
\&{return};\6
\4${}\}{}$\2\6
\4${}\}{}$\2\6
\&{while} ${}(\\{limit}\E\\{buffer}){}$;\par
\U12.\fi

\M15. \B\X15:Move \\{buffer} and \\{limit} to \\{change\_buffer} and \\{change%
\_limit}\X${}\E{}$\6
${}\{{}$\1\6
${}\\{change\_limit}\K\\{change\_buffer}-\\{buffer}+\\{limit};{}$\6
${}\\{strncpy}(\\{change\_buffer},\39\\{buffer},\39\\{limit}-\\{buffer}+%
\T{1});{}$\6
\4${}\}{}$\2\par
\Us12\ET16.\fi
!endgroup
\endgroup
\vfil\eject

\def\runninghead{APPENDIX D --- FINAL DOCUMENT}
\section Appendix D.
Here's what Appendix~C looks like when typeset. Appendix G contains the
entire context.

\M12. Procedure \\{prime\_the\_change\_buffer} sets \\{change\_buffer} in
preparation
for the next matching operation. Since blank lines in the change file are
not used for matching, we have $(\\{change\_limit}\E\\{change\_buffer}\W\R%
\\{changing})$
if and only if the change file is exhausted. This procedure is called only
when \\{changing} is 1; hence error messages will be reported correctly.

\Y\B\4\X4:Functions\X${}\mathrel+\E{}$\6
\\{prime\_the\_change\_buffer}(\,)\1\1\2\2\6
${}\{{}$\1\6
${}\\{change\_limit}\K\\{change\_buffer}{}$;\C{ this value is used if the
change file ends }\6
\X13:Skip over comment lines in the change file; \&{return} if end of file\X;\6
\X14:Skip to the next nonblank line; \&{return} if end of file\X;\6
\X15:Move \\{buffer} and \\{limit} to \\{change\_buffer} and \\{change\_limit}%
\X;\6
\4${}\}{}$\2\par
\fi

\M13. While looking for a line that begins with \.{@x} in the change file,
we allow lines that begin with \.{@}, as long as they don't begin with
\.{@y}, \.{@z} or \.{@i} (which would probably indicate that the
change file is fouled up).

\Y\B\4\X13:Skip over comment lines in the change file; \&{return} if end of
file\X${}\E{}$\6
\&{while} (\T{1})\5
${}\{{}$\1\6
${}\\{change\_line}\PP;{}$\6
\&{if} ${}(\R\\{input\_ln}(\\{change\_file})){}$\1\5
\&{return};\2\6
\&{if} ${}(\\{limit}<\\{buffer}+\T{2}){}$\1\5
\&{continue};\2\6
\&{if} ${}(\\{buffer}[\T{0}]\I\.{'@'}){}$\1\5
\&{continue};\2\6
\&{if} (\\{isupper}(\\{buffer}[\T{1}]))\1\5
${}\\{buffer}[\T{1}]\K\\{tolower}(\\{buffer}[\T{1}]);{}$\2\6
\&{if} ${}(\\{buffer}[\T{1}]\E\.{'x'}){}$\1\5
\&{break};\2\6
\&{if} ${}(\\{buffer}[\T{1}]\E\.{'y'}\V\\{buffer}[\T{1}]\E\.{'z'}\V\\{buffer}[%
\T{1}]\E\.{'i'}){}$\5
${}\{{}$\1\6
${}\\{loc}\K\\{buffer}+\T{2};{}$\6
\\{err\_print}(\.{"!\ This\ should\ only\ }\)\.{appear\ after\ a\ @x"});\6
\4${}\}{}$\2\6
\4${}\}{}$\2\par
\U12.\fi

\M14. Here we are looking at lines following the \.{@x}.

\Y\B\4\X14:Skip to the next nonblank line; \&{return} if end of file\X${}\E{}$\6
\&{do}\6
${}\{{}$\1\6
${}\\{change\_line}\PP;{}$\6
\&{if} ${}(\R\\{input\_ln}(\\{change\_file})){}$\5
${}\{{}$\1\6
\\{err\_print}(\.{"!\ Change\ file\ ended}\)\.{\ after\ @x"});\6
\&{return};\6
\4${}\}{}$\2\6
\4${}\}{}$\2\6
\&{while} ${}(\\{limit}\E\\{buffer}){}$;\par
\U12.\fi

\M15. \B\X15:Move \\{buffer} and \\{limit} to \\{change\_buffer} and \\{change%
\_limit}\X${}\E{}$\6
${}\{{}$\1\6
${}\\{change\_limit}\K\\{change\_buffer}-\\{buffer}+\\{limit};{}$\6
${}\\{strncpy}(\\{change\_buffer},\39\\{buffer},\39\\{limit}-\\{buffer}+%
\T{1});{}$\6
\4${}\}{}$\2\par
\Us12\ET16.\fi

\vfil\eject
\rightskip=0pt % get out of C mode (cf. \B)
\sfcode`;=1500 \pretolerance 200 \hyphenpenalty 50 \exhyphenpenalty 50

\def\runninghead{APPENDIX E --- MACROS FOR FORMATTING}
\section Appendix E: The \.{cwebmac.tex} file.
This is the file that extends ``plain \TeX'' format in order to support the
features needed by the output of \.{WEAVE}.

\vskip6pt
\begingroup \def\tt{\eighttt} \baselineskip9pt
\def\printmacs{\input cwebmac}
\verbatim
!printmacs
!endgroup
\endgroup
\vfill\eject

\def\runninghead{APPENDIX F --- NOTES ON FORMATTING}
\section Appendix F: How to use \.{CWEB} macros.
The macros in \.{cwebmac} make it possible to produce a variety of formats
without editing the output of \.{CWEAVE}, and the purpose of this appendix
is to explain some of the possibilities.

\def\point#1.{\yskip\indent#1.\quad\ignorespaces}

\point 1. Four fonts have been declared in addition to the standard fonts of
\.{PLAIN} format: You can say `\.{\{\\mc UNIX\}}' to get {\mc UNIX} in
medium-size caps; you can say `\.{\{\\sc STUFF\}}' to get {\sc STUFF}
in small caps; and you can select the largish fonts \.{\\titlefont}
and \.{\\ttitlefont} in the title of your document, where \.{\\ttitlefont}
is a typewriter style of type. There are macros \.{\\UNIX} and \.{\\Cee}
to refer to \UNIX\ and \Cee\ with medium-size caps.

\point 2. When you mention an identifier in \TeX\ text, you normally call
it `\.{|identifier|}'. But you can also say `\.{\\\\\{identifier\}}'. The
output will look the same in both cases, but the second alternative
doesn't put \\{identifier} into the index, since
it bypasses \.{WEAVE}'s translation from \Cee\ mode. In the second
case you have to put a backslash before each underline character
in the identifier.

\point 3. To get typewriter-like type, as when referring to `\.{WEB}', you
can use the `\.{\\.}' macro (e.g., `\.{\\.\{WEB\}}'). In the argument to
this macro you should insert an additional backslash before the symbols
listed as `special string characters' in the index to \.{WEAVE}, i.e.,
before backslashes and dollar signs and the like.
A `\.{\\\ }' here will result in the visible space symbol; to get an
invisible space following a control sequence you can say `\.{\{\ \}}'.
If the string is long, you can break it up into substrings that
are separated by `\.{\\)}'; the latter gives a discretionary backslash
if \TeX\ has to break a line here.

\point 4. The three control sequences \.{\\pagewidth}, \.{\\pageheight},
and \.{\\fullpageheight} can be redefined in the limbo section at the
beginning of your \.{WEB} file, to change the dimensions of each page.
The default settings
$$\lpile{\.{\\pagewidth=6.5in}\cr
  \.{\\pageheight=8.7in}\cr
  \.{\\fullpageheight=9in}\cr}$$
were used to prepare the present report; \.{\\fullpageheight} is
\.{\\pageheight} plus room for the additional heading and page numbers at
the top of each page. If you change any of these quantities, you should
call the macro \.{\\setpage} immediately after making the change.

\point 5. The \.{\\pageshift} macro defines an amount by which right-hand
pages (i.e., odd-numbered pages) are shifted right with respect to
left-hand (even-numbered) ones. By adjusting this amount you may be
able to get two-sided output in which the page numbers line up on
opposite sides of each sheet.

\point 6. The \.{\\title} macro will appear at the top of each page
in small caps; it is the job name unless redefined.

\point 7. The first page usually is assigned page
number 1. To start on page 16, with contents
on page 15, say this: `\.{\\def\\contentspagenumber\{15\}}
\.{\\pageno=\\contentspagenumber} \.{\\advance\\pageno by 1}'.

\point 8. The macro \.{\\iftitle} will suppress the header line if it is
defined by `\.{\\titletrue}'. The normal value is \.{\\titlefalse}
except for the table of contents; thus, the contents
page is usually unnumbered.

Two macros are provided to give flexibility to the table of
contents: \.{\\topofcontents} is invoked just before the contents
info is read, and \.{\\botofcontents} is invoked just after.
Here's a typical definition, taken from the original \.{WEB} manual:
$$\lpile{\.{\\def\\topofcontents\{\\null\\vfill}\cr
  \.{ { }\\titlefalse \% include headline on the contents page}\cr
  \.{ { }\\def\\rheader\{\\mainfont The \{\\tt WEAVE\}{ }processor\\hfil\}}\cr
  \.{ { }\\centerline\{\\titlefont The \{\\ttitlefont WEAVE\}{ }processor\}}\cr
  \.{ { }\\vskip 15pt \\centerline\{(Version 2.5)\}{ }\\vfill\}}\cr}$$
Redefining \.{\\rheader}, which is the headline for right-hand pages,
suffices in this case to put the desired information at the top of the
contents page.

\point 9. Data for the table of contents is written to a file that
is read after the indexes have been \TeX ed; there's one line of data
for every starred section. The file \.{common.toc} might look like this:
$$\lpile{\.{\\Z \{{ }Introduction\}\{1\}\{2\}}\cr
  \.{\\Z \{{ }The character set\}\{5\}\{3\}}\cr}$$
and so on. The \.{\\topofcontents} macro could
redefine \.{\\Z} so that the information appears in any desired format.

\point 10. Sometimes it is necessary or desirable to divide the output of
\.{WEAVE} into subfiles that can be processed separately. For example,
the listing of \TeX\ runs to more than 500 pages, and that is enough to
exceed the capacity of many printing devices and/or their software.
When an extremely large job isn't cut into smaller pieces, the entire
process might be spoiled by a single error of some sort, making it
necessary to start everything over.

Here's a safe way to break a woven file into three parts:
Say the pieces are $\alpha$,
$\beta$, and $\gamma$, where each piece begins with a starred section.
All macros should be defined in the opening limbo section of $\alpha$,
and copies of this \TeX\ code should be placed at the
beginning of $\beta$ and of $\gamma$. In order to process the parts
separately, we need to take care of two things: The starting page
numbers of $\beta$ and $\gamma$ need to be set up properly, and
the table of contents data from all three runs needs to be
accumulated.

The \.{webmac} macros include two control sequences \.{\\contentsfile} and
\.{\\readcontents} that facilitate the necessary processing.  We include
`\.{\\def\\contentsfile\{CONT1\}}' in the limbo section of $\alpha$, and
we include `\.{\\def\\contentsfile\{CONT2\}}' in the limbo section of
$\beta$; this causes \TeX\ to write the contents data for $\alpha$ and $\beta$
into \.{CONT1.TEX} and \.{CONT2.TEX}. Now in $\gamma$ we say
$$\.{\\def\\readcontents\{\\input CONT1 \\input CONT2 \\input CONTENTS\}};$$
this brings in the data from all three pieces, in the proper order.

However, we still need to solve the page-numbering problem. One way to
do it is to include the following in the limbo material for $\beta$:
$$\lpile{\.{\\message\{Please type the last page number of part 1: \}}\cr
  \.{\\read -1 to \\temp \\pageno=\\temp \\advance\\pageno by 1}\cr}$$
Then you simply provide the necessary data when \TeX\ requests
it; a similar construction is used at the beginning of $\gamma$.

This method can, of course, be used to divide a woven file into
any number of pieces.

\point 11. Sometimes it is nice to include things in the index that are
typeset in a special way. For example, we might want to have an
index entry for `\TeX'. \.{WEAVE} provides two simple ways to
typeset an index entry (unless the entry is an identifier or a reserved word):
`\.{@\^}' gives roman type, and `\.{@.}' gives typewriter type.
But if we try to typeset `\TeX' in roman type by saying, e.g.,
`\.{@\^\\TeX@>}', the backslash character gets in the way,
and this entry wouldn't appear in the index with the T's.

The solution is to use the `\.{@:}' feature, declaring a macro that
simply removes a sort key as follows:
$$\.{\\def\\9\#1\{\}}$$
Now you can say, e.g., `\.{@:TeX\}\{\\TeX@>}' in your \.{WEB} file; \.{WEAVE}
puts it into the index alphabetically, based on the sort key, and
produces the macro call `\.{\\9\{TeX\}\{\\TeX\}}' which will ensure that
the sort key isn't printed.

A similar idea can be used to insert hidden material into section
names so that they are alphabetized in whatever way you might wish.
Some people call these tricks ``special refinements''; others call
them ``kludges''.

\point 12. The control sequence \.{\\secno} is set to the number of the
section being typeset.

\point 13. If you want to list only the sections that have changed,
together with the index, put the command `\.{\\let\\maybe=\\iffalse}' in
the limbo section before the first section of your \.{WEB} file. It's
customary to make this the first change in your change file.

\point 14. To get output in languages other than English, redefine the
macros \.{\\A}, \.{\\ET}, \.{\\Q}, \.{\\U},
\.{\\ch}, \.{\\fin}, and \.{\\con}. \.{CWEAVE} itself need not be changed.

\point 15. All accents and special text symbols of plain \TeX\ format
will work in \.{CWEB} documents just as they are described in
Chapter~9 of {\sl The \TeX book}, with one exception.
The dot accent (normally \.{\\.}) must be typed \.{\\:} instead.

\vfill\end
