% -*- coding: utf-8 -*-
\input plain-t1

Here are some characters: $\alpha \Gamma$ æ Æ \'x

Ǎ Ǹ ă ŭ ā ē ü ł Ł ý ß Ş Ģ \c{g} ø ǽ Ǽ Ů ů ¡ ¿ ę Ą Ǫ ǫ Ų ų Į į

ĿL Ŀl ŀl

Český Krumlov (německy Böhmisch Krumau, popřípadě Krummau) je okresní
město v Jihočeském kraji, zhruba 25 km jižně od Českých
Budějovic. Rozkládá se pod hřebenem Blanského lesa a protéká jím řeka
Vltava. Jedná se o významné turistické centrum Jižních Čech.
Středověké centrum města, které obklopuje meandry Vltavy, je od roku
1963 městskou památkovou rezervací a od roku 1992 je zapsáno na
seznamu světového dědictví UNESCO. V roce 2003 bylo městskou
památkovou zónou vyhlášeno předměstí Plešivec (jižně od historického
jádra).

Český Krumlov é uma cidade República Checa na lista da UNESCO como
Patrimônio da Humanidade. Se encontra na Boêmia do Sul (região), é a
capital antiga da região de Rosenberg, a nobreza mais rica e influente
do país. A construção da cidade e seu Castelo começou no Século
XIII. A população da cidade em 2005 era de 13942 habitantes, e a área
de uns 22 km².

Český Krumlov (13.942 abitanti) è una città della Boemia meridionale,
in Repubblica Ceca, molto conosciuta per la raffinata architettura del
centro storico e per il Castello. Era conosciuta come Krumau fino alla
Seconda guerra mondiale quando alla fine furono espulsi gli abitanti
di lingua tedesca.  Český Krumlov letteralmente significa ``Krumlov
Ceca (Boema)''; ne esiste infatti anche una morava.

\UseUnicodeCharacter{00C8}

\bye
