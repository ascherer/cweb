\input cwebmac
\datethis
\nocon

\M{1}This is a quick program to find all canonical forms of reflection networks
for small $n$.

Well, when I wrote that paragraph I believed it, but subsequently I have added
lots of bells and whistles because I wanted to compute more stuff. At
present this code determines the number $B_n$ of equivalence classes of
reflection networks (i.e., irredundant primitive sorting networks);
also the number of weak equivalence classes, either with ($C_{n+1}$) or without
($D_{n+1}$) anti-isomorphism; and the number of preweak equivalence
classes ($E_{n+1}$), which is the number of simple arrangements of
$n+1$ pseudolines in a projective plane. For each representative of
$D_{n+1}$ it also computes the ``score,'' which is the number of ways
to add another pseudoline crossing the network.

If compiled without the \PB{\.{NOPRINT}} switch, each member of $B_n$ is
printed
as a string of transposition numbers, generated in
lexicographic order. This is followed by \.* if the string is also
a representative of $C_{n+1}$ when prefixed by $01\ldots n$. And if the string
is also a representative of $D_{n+1}$, you also get the score in brackets,
followed by \.\# if it is a representative of $E_{n+1}$. If not a
representative of $D_{n+1}$, the symbol \.> is printed followed by the
string of an anti-equivalent network.

If compiled with the \PB{\.{DEBUG}} switch, you also get intermediate output
about
the backtrack tree and the networks generated while searching for
anti-equivalence and preweak equivalence.

I wrote this program to allow $n$ up to 10; but integer overflow will
surely occur in $B_{10}\approx2\times10^{10}$, if I ever get a computer fast
enough to run that case. When $n=7$, this program took 48 seconds to run, on
January 12, 1991; the running time for $n=6$ was 1 second, and
for $n=8$ it was 57 minutes. Therefore I made a stripped-down version
to enumerate only $B_n$ when $n=9$.

\Y\B\8\#\&{include} \.{<stdio.h>}\par
\fi

\M{2}There's an array $a[1\mathrel{.\,.}n]$ containing $k$ inversions; an
index \PB{\|j} showing where we are going to try to reduce the inversions
by swapping \PB{\|a[\|j]} with \PB{$\|a[\|j+\T{1}]$}; and two arrays for
backtracking.
At choice-level \PB{\|l} we set \PB{\|t[\|l]} to the current \PB{\|j} value,
and
we also set \PB{\|c[\|l]} to 1 if we swapped, 0 if we didn't.

\Y\B\4\D$\\{swap}(\|j)$ \6
${}\{{}$\5
\1\&{int} \\{tmp}${}\K\|a[\|j]{}$;\5
${}\|a[\|j]\K\|a[\|j+\T{1}]{}$;\5
${}\|a[\|j+\T{1}]\K\\{tmp}{}$;\5
${}\}{}$\2\par
\B\4\D$\\{npairs}$ \5
\T{120}\C{ should be greater than $2{n+1\choose2}$ }\par
\B\4\D$\\{ncycle}$ \5
\T{240}\C{ should be greater than $4{n+1\choose2}$ }\par
\Y\B\4\X2:Global variables\X${}\E{}$\6
\&{int} \|n;\C{ number of elements to be reflected }\6
\&{int} \|a[\T{10}];\C{ array that shows progress }\6
\&{int} \|k;\C{ number of inversions yet to be removed }\6
\&{int} \|j;\C{ current place in array }\6
\&{int} \|l;\C{ current choice level }\6
\&{int} \|c[\\{npairs}];\C{ code for choices made }\6
\&{int} \|t[\\{npairs}];\C{ \PB{\|j} values where choices were made }\6
\&{int} \|i${},{}$ \\{ii}${},{}$ \\{iii};\C{ general-purpose indices }\6
\&{int} \\{bn}${},{}$ \\{cn}${},{}$ \\{dn}${},{}$ \\{en};\C{ counters for
$B_n$, $C_{n+1}$, $D_{n+1}$, $E_{n+1}$ }\6
\&{int} \\{smin}${},{}$ \\{smax};\C{ counters for ``scores'' }\6
\&{float} \\{stot};\C{ grand total of scores }\par
\As8\ET13.
\U3.\fi

\M{3}The value of \PB{\|n} is supposed to be an argument.

\Y\B\4\D$\\{abort}(\|s)$ \6
${}\{{}$\5
\1${}\\{fprintf}(\\{stderr},\39\|s){}$;\5
\\{exit}(\T{1});\5
${}\}{}$\2\par
\Y\B\X2:Global variables\X\7
${}\\{main}(\\{argc},\39\\{argv}){}$\1\1\6
\&{int} \\{argc};\C{ number of args }\6
\&{char} ${}{*}{*}\\{argv}{}$;\C{ the args }\2\2\6
${}\{{}$\1\6
\&{if} ${}(\\{argc}\I\T{2}){}$\1\5
\\{abort}(\.{"Usage:\ reflect\ n\\n"});\2\6
\&{if} ${}(\\{sscanf}(\\{argv}[\T{1}],\39\.{"\%d"},\39{\AND}\|n)\I\T{1}\V\|n<%
\T{2}\V\|n>\T{10}){}$\1\5
\\{abort}(\.{"n\ should\ be\ in\ the\ }\)\.{range\ 2..10!\\n"});\2\6
\X4:Initialize\X;\6
\X5:Run through all canonical reflection networks\X;\6
${}\\{printf}(\.{"B=\%d,\ C=\%d,\ D=\%d,\ E}\)\.{=\%d\\n"},\39\\{bn},\39\\{cn},%
\39\\{dn},\39\\{en});{}$\6
${}\\{printf}(\.{"scores\ min=\%d,\ max=}\)\.{\%d,\ mean=\%.1f\\n"},\39%
\\{smin},\39\\{smax},\39\\{stot}/{}$(\&{float}) \\{dn});\6
\4${}\}{}$\2\par
\fi

\M{4}\B\X4:Initialize\X${}\E{}$\6
\&{for} ${}(\|j\K\T{1};{}$ ${}\|j\Z\|n;{}$ ${}\|j\PP){}$\1\5
${}\|a[\|j]\K\|n+\T{1}-\|j;{}$\2\6
${}\|k\K\|n*(\|n-\T{1}){}$;\5
${}\|k\MRL{{/}{\K}}\T{2};{}$\6
${}\|c[\T{0}]\K\T{0}{}$;\C{ a convenient sentinel }\6
${}\|l\K\T{1};{}$\6
${}\|j\K\|n;{}$\6
${}\\{bn}\K\\{cn}\K\\{dn}\K\\{en}\K\\{smax}\K\T{0};{}$\6
${}\\{stot}\K\T{0.0};{}$\6
${}\\{smin}\K\T{1000000000}{}$;\par
\U3.\fi

\M{5}\B\X5:Run through all canonical reflection networks\X${}\E{}$\6
\4\\{moveleft}:\5
${}\|j\MM;{}$\6
\4\\{loop}:\6
\&{if} ${}(\|j\E\T{0}){}$\5
${}\{{}$\1\6
\&{if} ${}(\|k\E\T{0}){}$\1\5
\X7:Print a solution\X;\2\6
\X6:Backtrack, either going to \PB{\\{loop}} or to \PB{\\{finished}} when all
possibilities are exhausted\X;\6
\4${}\}{}$\2\6
\&{if} ${}(\|a[\|j]<\|a[\|j+\T{1}]){}$\1\5
\&{goto} \\{moveleft};\2\6
${}\|t[\|l]\K\|j;{}$\6
${}\|c[\|l\PP]\K\T{0};{}$\6
\&{goto} \\{moveleft};\6
\4\\{finished}:\5
;\par
\U3.\fi

\M{6}\B\X6:Backtrack, either going to \PB{\\{loop}} or to \PB{\\{finished}}
when all possibilities are exhausted\X${}\E{}$\6
\&{while} ${}(\|c[\MM\|l]){}$\5
${}\{{}$\1\6
${}\|j\K\|t[\|l];{}$\6
\\{swap}(\|j);\6
${}\|k\PP;{}$\6
\4${}\}{}$\2\6
\&{if} ${}(\|l\E\T{0}){}$\1\5
\&{goto} \\{finished};\2\6
${}\|j\K\|t[\|l];{}$\6
${}\|c[\|l\PP]\K\T{1};{}$\6
\\{swap}(\|j);\6
${}\|k\MM;{}$\6
\&{if} ${}(\PP\|j\E\|n){}$\1\5
${}\|j\MM;{}$\2\6
\&{goto} \\{loop};\par
\U5.\fi

\M{7}\B\X7:Print a solution\X${}\E{}$\6
${}\{{}$\6
\8\#\&{ifdef} \.{DEBUG}\1\6
\&{for} ${}(\|i\K\T{1};{}$ ${}\|i<\|l;{}$ ${}\|i\PP){}$\1\5
${}\\{putchar}(\.{'0'}+\|c[\|i]);{}$\2\6
\\{putchar}(\.{':'});\6
\8\#\&{endif}\6
\8\#\&{ifndef} \.{NOPRINT}\6
\&{for} ${}(\|i\K\T{1};{}$ ${}\|i<\|l;{}$ ${}\|i\PP){}$\1\6
\&{if} (\|c[\|i])\1\5
${}\\{putchar}(\.{'0'}-\T{1}+\|t[\|i]);{}$\2\2\6
\8\#\&{endif}\6
\X9:Check if it gives a new CC system on \PB{$\|n+\T{1}$} elements\X;\6
\8\#\&{ifndef} \.{NOPRINT}\6
\\{putchar}(\.{'\\n'});\6
\8\#\&{endif}\6
${}\\{bn}\PP;{}$\6
\4${}\}{}$\2\par
\U5.\fi

\M{8}Here's part of the program I wrote after getting the above to work.
The idea is to see if the almost-canonical form for an (\PB{$\|n+%
\T{1}$})-element
network is weakly equivalent to any lexicographically smaller
almost-canonical forms. If not, we print an asterisk, because it
represents a new weak equivalence class.

The forms are kept in locations \PB{\|r} through \PB{$\|r+\|n(\|n+\T{1})/\T{2}-%
\T{1}$} of array \PB{\|b},
which starts out like \PB{\|t} but with the transpositions 1, 2, \dots,~$n$
prefaced. End-around shifts are performed (advancing \PB{\|r} by~1 each time)
until the original form appears again.

\Y\B\4\X2:Global variables\X${}\mathrel+\E{}$\6
\&{int} \|b[\\{ncycle}];\C{ larger array used for testing weak equivalence }\6
\&{int} \|r${},{}$ \\{rr};\C{ the first and last active locations in \PB{\|b} }%
\6
\&{int} \|d[\\{npairs}];\C{ copy of the present network }\6
\&{int} \\{rrr};\C{ $n+1\choose2$ }\par
\fi

\M{9}\B\X9:Check if it gives a new CC system on \PB{$\|n+\T{1}$} elements\X${}%
\E{}$\6
\&{for} ${}(\\{rr}\K\T{0};{}$ ${}\\{rr}<\|n;{}$ ${}\\{rr}\PP){}$\1\5
${}\|b[\\{rr}]\K\\{rr}+\T{1};{}$\2\6
\&{for} ${}(\|i\K\T{1};{}$ ${}\|i<\|l;{}$ ${}\|i\PP){}$\1\6
\&{if} (\|c[\|i])\5
${}\{{}$\1\6
${}\|b[\\{rr}]\K\|d[\\{rr}]\K\|t[\|i];{}$\6
${}\\{rr}\PP;{}$\6
\4${}\}{}$\2\2\6
${}\|d[\\{rr}]\K\T{1}{}$;\C{ sentinel }\6
${}\\{rrr}\K\\{rr};{}$\6
${}\|r\K\T{0};{}$\6
\&{while} (\T{1})\5
${}\{{}$\1\6
\X10:Shift the first transposition to the other end\X;\6
\&{if} ${}(\|b[\|r]\E\T{1}){}$\1\5
\X11:Test lexicographic order; \PB{\&{break}} if equal or less\X;\2\6
\4${}\}{}$\2\par
\U7.\fi

\M{10}\B\X10:Shift the first transposition to the other end\X${}\E{}$\6
$\|j\K\|n-\|b[\|r\PP];{}$\6
\&{for} ${}(\|i\K\\{rr}\PP;{}$ ${}\|b[\|i-\T{1}]<\|j;{}$ ${}\|i\MM){}$\1\5
${}\|b[\|i]\K\|b[\|i-\T{1}];{}$\2\6
${}\|b[\|i]\K\|j+\T{1}{}$;\par
\U9.\fi

\M{11}\B\X11:Test lexicographic order; \PB{\&{break}} if equal or less\X${}%
\E{}$\6
${}\{{}$\1\6
${}\|b[\\{rr}]\K\T{0}{}$;\C{ sentinel, is less than the 1 we put in \PB{\|d} }\6
\&{for} ${}(\|i\K\|r+\|n;{}$ ${}\|b[\|i]\E\|d[\|i-\|r];{}$ ${}\|i\PP){}$\1\5
;\2\6
\&{if} ${}(\|b[\|i]<\|d[\|i-\|r]){}$\5
${}\{{}$\1\6
\&{if} ${}(\|i\E\\{rr}){}$\5
${}\{{}$\C{ total equality }\6
\8\#\&{ifndef} \.{NOPRINT}\1\6
\\{putchar}(\.{'*'});\6
\8\#\&{endif}\6
${}\\{cn}\PP;{}$\6
\X12:Make the big test for pre-weak equivalence\X;\6
\4${}\}{}$\2\6
\&{break};\6
\4${}\}{}$\2\6
\4${}\}{}$\2\par
\U9.\fi

\M{12}Well, after I got that going I couldn't resist continuing until I had
all simple arrangements of pseudolines enumerated. That requires looking at
another $n+1\choose2$ cases to see if they are weakly equivalent to anything
seen before.

And, surprise, it also meant testing for anti-isomorphism.

\Y\B\4\X12:Make the big test for pre-weak equivalence\X${}\E{}$\6
\X14:Reset \PB{\|b} to a double cycle\X;\6
\X15:Test the reverse of \PB{\|b} for weak equivalence; \PB{\&{goto} \\{done}}
if weakly equivalent to a previous case\X;\6
\X22:Compute the score for this weak equivalence/antiequivalence class rep\X;\6
\&{for} ${}(\|r\K\T{0};{}$ ${}\|r<\\{rrr};{}$ ${}\|r\PP){}$\5
${}\{{}$\1\6
\X20:Move the ``pole'' into the cell preceding the first transposition module%
\X;\6
\&{for} ${}(\\{ref}\K\T{0};{}$ ${}\\{ref}<\T{2};{}$ ${}\\{ref}\PP){}$\5
${}\{{}$\1\6
\&{if} ${}(\\{ref}\E\T{0}){}$\1\6
\&{for} ${}(\|i\K\T{0};{}$ ${}\|i<\\{rrr};{}$ ${}\|i\PP){}$\1\5
${}\|y[\|i]\K\|x[\|i];{}$\2\2\6
\&{else}\1\5
\X16:Replace the present \PB{\|x} by the reverse of \PB{\|y}\X;\2\6
\X17:If the new network is weakly equivalent to a lexicographically smaller
one, \PB{\&{goto} \\{done}}\X;\6
\4${}\}{}$\2\6
\4${}\}{}$\2\6
\8\#\&{ifndef} \.{NOPRINT}\6
\\{putchar}(\.{'\#'});\C{ a new preweak class, not related to anything earlier
}\6
\8\#\&{endif}\6
${}\\{en}\PP;{}$\6
\4\\{done}:\5
;\par
\U11.\fi

\M{13}For this part of the program we use an array \PB{\|x} analogous to \PB{%
\|b}; also
variables \PB{\|s} and \PB{\\{ss}} analogous to \PB{\|r} and~\PB{\\{rr}}; also
an array \PB{\|e} analogous
to \PB{\|d}.

\Y\B\4\X2:Global variables\X${}\mathrel+\E{}$\6
\&{int} \|x[\\{ncycle}];\C{ network to be tested for weak equivalence }\6
\&{int} \|m;\C{ largest element in \PB{\|x} so far }\6
\&{int} \|y[\\{npairs}];\C{ elements to be carried around to the right as \PB{%
\|x} is formed }\6
\&{int} \\{jj};\C{ the number of elements in \PB{\|y} }\6
\&{int} \|s${},{}$ \\{ss};\C{ the active region of \PB{\|x} }\6
\&{int} \|e[\\{npairs}];\C{ starting point }\6
\&{int} \\{rep};\C{ number of repetitions }\6
\&{int} \\{ref};\C{ number of reflections }\par
\fi

\M{14}At this point \PB{$\|i-\|r$} points just past the end of the \PB{\|d}
data, and
the first \PB{\|n} entries of \PB{\|b} are still equal to 1, 2, \dots,~\PB{%
\|n}.
The network we construct here is not necessarily in canonical form.

\Y\B\4\X14:Reset \PB{\|b} to a double cycle\X${}\E{}$\6
$\\{rr}\K\|i-\|r;{}$\6
\&{for} ${}(\|i\K\|n;{}$ ${}\|i<\\{rr};{}$ ${}\|i\PP){}$\1\5
${}\|b[\|i]\K\|d[\|i];{}$\2\6
\&{for} ( ; ${}\|i<\\{rr}+\\{rr};{}$ ${}\|i\PP){}$\1\5
${}\|b[\|i]\K\|n+\T{1}-\|b[\|i-\\{rr}]{}$;\2\par
\U12.\fi

\M{15}One nice thing is that reflection and turning upside down preserve
canonicity when we do both simultaneously.

\Y\B\4\X15:Test the reverse of \PB{\|b} for weak equivalence; \PB{\&{goto} %
\\{done}} if weakly equivalent to a previous case\X${}\E{}$\6
\&{for} ${}(\|i\K\T{0};{}$ ${}\|i<\\{rrr};{}$ ${}\|i\PP){}$\1\5
${}\|x[\\{rrr}-\T{1}-\|i]\K\|n+\T{1}-\|b[\|i];{}$\2\6
${}\|s\K\T{0}{}$;\5
${}\\{ss}\K\\{rrr};{}$\6
\&{while} ${}(\|x[\|s]>\T{1}){}$\1\5
\X19:End-around shift \PB{\|x}\X;\2\6
\&{for} ${}(\|i\K\|s+\|n;{}$ ${}\|i<\\{ss};{}$ ${}\|i\PP){}$\1\5
${}\|e[\|i-\|s]\K\|x[\|i];{}$\2\6
${}\|e[\\{rrr}]\K\T{1}{}$;\C{ another sentinel }\6
\&{while} (\T{1}) $\{$ $\|x[\\{ss}]\K\T{0}{}$;\C{ sentinel }\6
\&{for} ${}(\|i\K\|s+\|n;{}$ ${}\|x[\|i]\E\|d[\|i-\|s];{}$ ${}\|i\PP){}$\1\5
;\2\6
\&{if} ${}(\|i\E\\{ss}){}$\1\5
\&{break};\C{ anti-isomorphic to itself }\2\6
\&{if} ${}(\|x[\|i]<\|d[\|i-\|s]){}$\5
${}\{{}$\C{ anti-isomorphic to previous guy }\6
\8\#\&{ifndef} \.{NOPRINT}\1\6
\\{putchar}(\.{'>'});\6
\&{for} ${}(\|i\K\|s+\|n;{}$ ${}\|i<\\{ss};{}$ ${}\|i\PP){}$\1\5
${}\\{putchar}(\|x[\|i]+\.{'0'}-\T{1});{}$\2\6
\8\#\&{endif}\6
\&{goto} \\{done};\6
\4${}\}{}$\2\6
\&{do} \X19:End-around shift \PB{\|x}\X \6
\&{while} ${}(\|x[\|s]>\T{1}){}$\1\5
;\2\6
${}\|x[\\{ss}]\K\T{0};{}$\6
\&{for} ${}(\|i\K\|s+\|n;{}$ ${}\|x[\|i]\E\|e[\|i-\|s];{}$ ${}\|i\PP){}$\1\5
;\2\6
\&{if} ${}(\|i\E\\{ss}){}$\1\5
\&{break};\C{ anti-isomorphic to some future guy }\2\6
$\}{}$\par
\U12.\fi

\M{16}\B\X16:Replace the present \PB{\|x} by the reverse of \PB{\|y}\X${}\E{}$\6
${}\{{}$\1\6
\&{for} ${}(\|i\K\T{0};{}$ ${}\|i<\\{rrr};{}$ ${}\|i\PP){}$\1\5
${}\|x[\\{rrr}-\T{1}-\|i]\K\|n+\T{1}-\|y[\|i];{}$\2\6
${}\|s\K\T{0}{}$;\5
${}\\{ss}\K\\{rrr};{}$\6
\&{while} ${}(\|x[\|s]>\T{1}){}$\1\5
\X19:End-around shift \PB{\|x}\X;\2\6
\8\#\&{ifdef} \.{DEBUG}\6
\\{putchar}(\.{'/'});\6
\X25:If debugging, print the active region of \PB{\|x}\X;\6
\8\#\&{endif}\6
\4${}\}{}$\2\par
\U12.\fi

\M{17}\B\X17:If the new network is weakly equivalent to a lexicographically
smaller one, \PB{\&{goto} \\{done}}\X${}\E{}$\6
\&{for} ${}(\|i\K\|s+\|n;{}$ ${}\|i<\\{ss};{}$ ${}\|i\PP){}$\1\5
${}\|e[\|i-\|s]\K\|x[\|i];{}$\2\6
\&{while} (\T{1}) $\{$ \X18:If the \PB{\|x} network is weakly equivalent to an
earlier one, \PB{\&{goto} \\{done}}; if weakly equivalent to the present one, %
\PB{\&{goto} \\{okay}}\X;\6
\&{do} \X19:End-around shift \PB{\|x}\X \6
\&{while} ${}(\|x[\|s]>\T{1}){}$\1\5
;\2\6
\X25:If debugging, print the active region of \PB{\|x}\X;\6
${}\|x[\\{ss}]\K\T{0}{}$;\C{ sentinel }\6
\&{for} ${}(\|i\K\|s+\|n;{}$ ${}\|x[\|i]\E\|e[\|i-\|s];{}$ ${}\|i\PP){}$\1\5
;\2\6
\&{if} ${}(\|i\E\\{ss}){}$\1\5
\&{break};\C{ now \PB{\|x} is back to its original state and we found nothing }%
\2\6
$\}$ \6
\4\\{okay}:\5
;\par
\U12.\fi

\M{18}\B\X18:If the \PB{\|x} network is weakly equivalent to an earlier one, %
\PB{\&{goto} \\{done}}; if weakly equivalent to the present one, \PB{\&{goto} %
\\{okay}}\X${}\E{}$\6
$\|x[\\{ss}]\K\T{0}{}$;\C{ sentinel }\6
\&{for} ${}(\|i\K\|s+\|n;{}$ ${}\|x[\|i]\E\|d[\|i-\|s];{}$ ${}\|i\PP){}$\1\5
;\2\6
\&{if} ${}(\|i\E\\{ss}){}$\1\5
\&{goto} \\{okay};\2\6
\&{if} ${}(\|x[\|i]<\|d[\|i-\|s]){}$\1\5
\&{goto} \\{done};\2\par
\U17.\fi

\M{19}\B\X19:End-around shift \PB{\|x}\X${}\E{}$\6
${}\{{}$\1\6
${}\|j\K\|n-\|x[\|s\PP];{}$\6
\&{for} ${}(\|i\K\\{ss}\PP;{}$ ${}\|x[\|i-\T{1}]<\|j;{}$ ${}\|i\MM){}$\1\5
${}\|x[\|i]\K\|x[\|i-\T{1}];{}$\2\6
${}\|x[\|i]\K\|j+\T{1};{}$\6
\4${}\}{}$\2\par
\Us15, 16\ETs17.\fi

\M{20}The only somewhat tricky operation comes in here. We use the fact that
the first `1' in a canonical network is always immediately followed by
2, \dots,~\PB{\|n}; reversing these, decreasing the previous by~1, and
increasing
the remaining by~1 takes that line around the pole. This operation might
require carrying some transpositions around from left to right.

\Y\B\4\X20:Move the ``pole'' into the cell preceding the first transposition
module\X${}\E{}$\6
\X24:If debugging, print the active region of \PB{\|b}\X;\6
${}\|s\K\T{0}{}$;\5
${}\\{ss}\K\\{rrr};{}$\6
${}\\{iii}\K\\{jj}\K\T{0};{}$\6
${}\|x[\T{0}]\K\|m\K\\{rep}\K\|b[\|r];{}$\6
${}\\{rr}\K\|r+\\{rrr};{}$\6
\&{for} ${}(\|i\K\|r+\T{1};{}$ ${}\|i<\\{rr};{}$ ${}\|i\PP){}$\5
${}\{{}$\1\6
${}\|j\K\|b[\|i]-\T{1};{}$\6
\X21:Insert the value \PB{$\|j+\T{1}$} canonically into \PB{\|x}\X;\6
\4${}\}{}$\2\6
\&{for} ${}(\|i\K\T{0};{}$ ${}\\{iii}<\\{rrr}-\T{1};{}$ ${}\|i\PP){}$\5
${}\{{}$\1\6
${}\|j\K\|n-\T{1}-\|y[\|i];{}$\6
\X21:Insert the value \PB{$\|j+\T{1}$} canonically into \PB{\|x}\X;\6
\4${}\}{}$\2\6
\X25:If debugging, print the active region of \PB{\|x}\X;\6
\&{while} ${}(\\{rep}\MM){}$\5
${}\{{}$\1\6
${}\|m\K\T{0};{}$\6
\&{for} ${}(\|i\K\T{0};{}$ ${}\|x[\|i]\I\T{1};{}$ ${}\|i\PP){}$\5
${}\{{}$\1\6
${}\|x[\|i]\MM;{}$\6
\&{if} ${}(\|x[\|i]>\|m){}$\1\5
${}\|m\K\|x[\|i];{}$\2\6
\4${}\}{}$\2\6
${}\\{iii}\K\|i-\T{1};{}$\6
${}\\{jj}\K\T{0};{}$\6
\&{for} ${}(\|j\K\|n-\T{1};{}$ ${}\|j\G\T{0};{}$ ${}\|j\MM){}$\1\6
\&{if} ${}(\|j\E\T{0}\W\|i\E\T{0}){}$\5
${}\{{}$\1\6
${}\|x[\T{0}]\K\|m\K\T{1};{}$\6
${}\\{iii}\K\T{0};{}$\6
\4${}\}{}$\2\6
\&{else}\1\5
\X21:Insert the value \PB{$\|j+\T{1}$} canonically into \PB{\|x}\X;\2\2\6
\&{for} ${}(\|i\MRL{+{\K}}\|n;{}$ ${}\|i<\\{rrr};{}$ ${}\|i\PP){}$\5
${}\{{}$\1\6
${}\|j\K\|x[\|i];{}$\6
\X21:Insert the value \PB{$\|j+\T{1}$} canonically into \PB{\|x}\X;\6
\4${}\}{}$\2\6
\&{for} ${}(\|i\K\T{0};{}$ ${}\\{iii}<\\{ss}-\T{1};{}$ ${}\|i\PP){}$\5
${}\{{}$\1\6
${}\|j\K\|n-\T{1}-\|y[\|i];{}$\6
\X21:Insert the value \PB{$\|j+\T{1}$} canonically into \PB{\|x}\X;\6
\4${}\}{}$\2\6
\X25:If debugging, print the active region of \PB{\|x}\X;\6
\4${}\}{}$\2\par
\U12.\fi

\M{21}We must carry over items that exceed \PB{\|m}, which denotes the maximum
value stored so far, because we want the first element of \PB{\|x[\T{0}]} to
remain
in place.

\Y\B\4\X21:Insert the value \PB{$\|j+\T{1}$} canonically into \PB{\|x}\X${}%
\E{}$\6
\&{if} ${}(\|j>\|m){}$\1\5
${}\|y[\\{jj}\PP]\K\|j;{}$\2\6
\&{else}\5
${}\{{}$\1\6
\&{if} ${}(\|j\E\|m){}$\1\5
${}\|m\PP;{}$\2\6
\&{for} ${}(\\{ii}\K\PP\\{iii};{}$ ${}\|x[\\{ii}-\T{1}]<\|j;{}$ ${}\\{ii}%
\MM){}$\1\5
${}\|x[\\{ii}]\K\|x[\\{ii}-\T{1}];{}$\2\6
${}\|x[\\{ii}]\K\|j+\T{1};{}$\6
\4${}\}{}$\2\par
\U20.\fi

\M{22}The score is computed in several passes, although I do know how to
do it in linear time. Since the \PB{\|x} array is currently unused, I store
in \PB{\|x[\|i]} the score for the cell following transposition \PB{\|i}.

\Y\B\4\X22:Compute the score for this weak equivalence/antiequivalence class
rep\X${}\E{}$\6
$\\{dn}\PP;{}$\6
${}\\{rr}\K\\{rrr}+\\{rrr};{}$\6
\&{for} ${}(\|i\K\T{0};{}$ ${}\|i<\\{rr};{}$ ${}\|i\PP){}$\1\5
${}\|x[\|i]\K\T{1};{}$\2\6
\&{for} ${}(\|j\K\T{2};{}$ ${}\|j\Z\|n;{}$ ${}\|j\PP){}$\1\5
\X23:Fill in the cell counts \PB{\|x[\|i]} for cases when \PB{$\|b[\|i]\K\|j$}%
\X;\2\6
${}\{{}$\5
\1\&{register} \&{int} \\{score}${}\K\T{0};{}$\7
\&{for} ${}(\|i\K\T{0};{}$ ${}\|i<\\{rr};{}$ ${}\|i\PP){}$\1\6
\&{if} ${}(\|b[\|i]\E\|n){}$\1\5
${}\\{score}\MRL{+{\K}}\|x[\|i];{}$\2\2\6
${}\\{stot}\MRL{+{\K}}{}$(\&{float}) \\{score};\6
\&{if} ${}(\\{score}>\\{smax}){}$\1\5
${}\\{smax}\K\\{score};{}$\2\6
\&{if} ${}(\\{score}<\\{smin}){}$\1\5
${}\\{smin}\K\\{score};{}$\2\6
\8\#\&{ifndef} \.{NOPRINT}\6
${}\\{printf}(\.{"\ [\%d]"},\39\\{score});{}$\6
\8\#\&{endif}\6
\4${}\}{}$\2\par
\U12.\fi

\M{23}As we fill the cell counts, we assume that \PB{\|x[\\{ii}]} is the
previous cell
having \PB{$\|b[\|i]\K\|j$}. We assume that \PB{$\|b[\|i]\E\|i+\T{1}$} for %
\PB{$\T{0}\Z\|i<\|n$}.

\Y\B\4\X23:Fill in the cell counts \PB{\|x[\|i]} for cases when \PB{$\|b[\|i]\K%
\|j$}\X${}\E{}$\6
${}\{{}$\5
\1\&{int} \\{acc}${}\K\T{0};{}$\6
\&{int} \|p;\C{ most recent \PB{\|x[\|i]} when \PB{$\|b[\|i]\K\|j-\T{1}$} }\7
${}\\{ii}\K\\{rr};{}$\6
\&{for} ${}(\|i\K\T{0};{}$ ${}\|i<\\{rr};{}$ ${}\|i\PP){}$\5
${}\{{}$\5
\1\&{register} \&{int} \\{delta}${}\K\|j-\|b[\|i];{}$\7
\&{if} ${}(\\{delta}\E\T{0}){}$\5
${}\{{}$\1\6
${}\|x[\\{ii}]\K\\{acc};{}$\6
${}\\{ii}\K\|i;{}$\6
${}\\{acc}\K\|p;{}$\6
\4${}\}{}$\2\6
\&{else} \&{if} ${}(\\{delta}\E\T{1}){}$\5
${}\{{}$\1\6
${}\|p\K\|x[\|i];{}$\6
${}\\{acc}\MRL{+{\K}}\|p;{}$\6
\4${}\}{}$\2\6
\4${}\}{}$\2\6
${}\|x[\\{ii}]\K\\{acc}+\|x[\\{rr}];{}$\6
\4${}\}{}$\2\par
\U22.\fi

\M{24}\B\X24:If debugging, print the active region of \PB{\|b}\X${}\E{}$\6
\8\#\&{ifdef} \.{DEBUG}\6
\\{printf}(\.{"\\n>"});\6
\&{for} ${}(\|m\K\|r;{}$ ${}\|m<\|r+\\{rrr};{}$ ${}\|m\PP){}$\1\5
${}\\{putchar}(\|b[\|m]+\.{'0'}-\T{1});{}$\2\6
\8\#\&{endif}\par
\U20.\fi

\M{25}\B\X25:If debugging, print the active region of \PB{\|x}\X${}\E{}$\6
\8\#\&{ifdef} \.{DEBUG}\6
\\{printf}(\.{"\\n\ \ "});\6
\&{for} ${}(\|m\K\|s;{}$ ${}\|m<\\{ss};{}$ ${}\|m\PP){}$\1\5
${}\\{putchar}(\|x[\|m]+\.{'0'}-\T{1});{}$\2\6
\8\#\&{endif}\par
\Us16, 17\ETs20.\fi

\inx
\fin
\end
