\catcode`@=11

\input utfplainmac

\message{Loading EC fonts}

\font\tenrm=ecrm1000 % roman text
\font\tctenrm=tcrm1000
% \font\sevenrm=ecrm0700
% \font\fiverm=ecrm0500

\font\tenbf=ecbx1000 % boldface extended
\font\tctenbf=tcbx1000
% \font\sevenbf=ecbx0700
% \font\fivebf=ecbx0500

\font\tentt=ectt1000 % typewriter
\font\tctentt=tctt1000

\font\tensl=ecsl1000 % slanted roman
\font\tctensl=tcsl1000

\font\tenit=ecti1000 % text italic
\font\tctenit=tcti1000

% \font\tenrm=ptmr8t % roman text
% \font\sevenrm=ptmr8t at 7pt
% \font\fiverm=ptmr8t at 5pt

% \font\tenbf=ptmb8t % boldface extended
% \font\sevenbf=ptmb8t at 7pt
% \font\fivebf=ptmb8t at 5pt

% \font\tentt=pcrr8t % typewriter

% \font\tensl=ptmro8t % slanted roman

% \font\tenit=ptmri8t % text italic

% \textfont0=\tenrm \scriptfont0=\sevenrm \scriptscriptfont0=\fiverm
% \textfont1=\teni \scriptfont1=\seveni \scriptscriptfont1=\fivei
% \textfont2=\tensy \scriptfont2=\sevensy \scriptscriptfont2=\fivesy
% \textfont3=\tenex \scriptfont3=\tenex \scriptscriptfont3=\tenex
% \textfont\itfam=\tenit
% \textfont\slfam=\tensl
% \textfont\bffam=\tenbf \scriptfont\bffam=\sevenbf
%   \scriptscriptfont\bffam=\fivebf
% \textfont\ttfam=\tentt

\def\rm{\fam\z@\let\tcfont\tctenrm\tenrm}
\def\it{\fam\itfam\let\tcfont\tctenit\tenit}
\def\sl{\fam\slfam\let\tcfont\tctensl\tensl}
\def\bf{\fam\bffam\let\tcfont\tctenbf\tenbf}
\def\tt{\fam\ttfam\let\tcfont\tctentt\tentt}

% set the font
\rm

\catcode`\@=11

% special characters
\chardef\pound="BF
\chardef\IJ="9C
\chardef\ij="BC
\chardef\L="8A
\chardef\l="AA
\chardef\DH="D0
\chardef\dh="F0
\chardef\TH="DE
\chardef\th="FE
\chardef\NG="8D
\chardef\ng="AD
\chardef\AA="C5
\chardef\aa="E5
\chardef\AE="C6
\chardef\ae="E6
\chardef\OE="D7
\chardef\oe="F7
\chardef\O="D8
\chardef\o="F8
\chardef\SS="DF
\chardef\ss="FF
\chardef\i="19
\chardef\j="1A
\let\DJ=\DH
\chardef\dj="9E

\def\@firstoftwo#1#2{#1}
\def\@secondoftwo#1#2{#2}
\def\@ifundefined#1{\expandafter\ifx\csname#1\endcsname\relax
  \expandafter\@firstoftwo\else\expandafter\@secondoftwo\fi}

%%% \make@ec@accent is syntactic sugar; for example
%%% \make@ec@accent\x{abc} is equivalent to
%%%
%%% \def\x#1{\@ifundefined{ec@abc@\detokenize{#1}}
%%%   {\csname ec@abc\endcsname{#1}}{\csname ec@abc@#1\endcsname}}}
%%%
%%% Thus a call like \x{y} looks whether \ec@abc@y is defined; if it
%%% is, then use it, otherwise resort to \ec@abc{y}, where \ec@abc is
%%% the general accent command. In this way we can define \x{y} to
%%% print a single character, for hyphenation purposes, for example.
\def\make@ec@accent#1#2{%
  \def#1##1{\@ifundefined{ec@#2@\detokenize{##1}}
    {\csname ec@#2\endcsname{##1}}{\csname ec@#2@##1\endcsname}}}
\make@ec@accent\`{grave}
\make@ec@accent\'{acute}
\make@ec@accent\^{circumflex}
\make@ec@accent\~{tilde}
\make@ec@accent\"{dieresis}
\make@ec@accent\H{doubleacute}
\make@ec@accent\r{ring}
\make@ec@accent\v{caron}
\make@ec@accent\u{breve}
\make@ec@accent\={macron}
\make@ec@accent\.{dotabove}
\make@ec@accent\c{cedilla}
\make@ec@accent\k{ogonek}

%%% Now we define the accents; for example \ec@grave is defined as it
%%% is \` in Plain TeX, except for the code point of the accent. But
%%% we define also \ec@grave@A to print just a character which will
%%% then participate to hyphenation and kerning. The same for all
%%% other characters which are available in T1 encoded fonts. In some
%%% special cases we provide also some complicated definition, to
%%% cover peculiar situation (like \c{g}, where the cedilla should go
%%% over the g).

% grave accent
\def\ec@grave#1{{\accent"0 #1}}
\chardef\ec@grave@A="C0
\chardef\ec@grave@a="E0
\chardef\ec@grave@E="C8
\chardef\ec@grave@e="E8
\chardef\ec@grave@I="CC
\chardef\ec@grave@i="EC
\chardef\ec@grave@O="D2
\chardef\ec@grave@o="F2
\chardef\ec@grave@U="D9
\chardef\ec@grave@u="F9

% acute accent
\def\ec@acute#1{{\accent"1 #1}}
\chardef\ec@acute@A="C1
\chardef\ec@acute@a="E1
\chardef\ec@acute@E="C9
\chardef\ec@acute@e="E9
\chardef\ec@acute@I="CD
\chardef\ec@acute@i="ED
\chardef\ec@acute@C="82
\chardef\ec@acute@c="A2
\chardef\ec@acute@L="88
\chardef\ec@acute@l="A8
\chardef\ec@acute@N="8B
\chardef\ec@acute@n="AB
\chardef\ec@acute@O="D3
\chardef\ec@acute@o="F3
\chardef\ec@acute@R="8F
\chardef\ec@acute@r="AF
\chardef\ec@acute@S="91
\chardef\ec@acute@s="B1
\chardef\ec@acute@U="DA
\chardef\ec@acute@u="FA
\chardef\ec@acute@Z="99
\chardef\ec@acute@z="B9

% circumflex accent
\def\ec@circumflex#1{{\accent"2 #1}}
\chardef\ec@circumflex@A="C2
\chardef\ec@circumflex@a="E2
\chardef\ec@circumflex@E="CA
\chardef\ec@circumflex@e="EA
\chardef\ec@circumflex@I="CE
\chardef\ec@circumflex@i="EE
\chardef\ec@circumflex@O="D4
\chardef\ec@circumflex@o="F4
\chardef\ec@circumflex@U="DB
\chardef\ec@circumflex@u="FB

% tilde accent
\def\ec@tilde#1{{\accent"3 #1}}
\chardef\ec@tilde@A="C3
\chardef\ec@tilde@a="E3
\chardef\ec@tilde@N="D1
\chardef\ec@tilde@n="F1
\chardef\ec@tilde@O="D5
\chardef\ec@tilde@o="F5

% dieresis
\def\ec@dieresis#1{{\accent"4 #1}}
\chardef\ec@dieresis@A="C4
\chardef\ec@dieresis@a="E4
\chardef\ec@dieresis@E="CB
\chardef\ec@dieresis@e="EB
\chardef\ec@dieresis@I="CF
\chardef\ec@dieresis@i="EF
\chardef\ec@dieresis@O="D6
\chardef\ec@dieresis@o="F6
\chardef\ec@dieresis@U="DC
\chardef\ec@dieresis@u="FC
\chardef\ec@dieresis@Y="98
\chardef\ec@dieresis@y="A8

% double acute (hungarian umlaut)
\def\ec@doubleacute#1{{\accent"5 #1}}
\chardef\ec@doubleacute@O="8E
\chardef\ec@doubleacute@o="AE
\chardef\ec@doubleacute@U="97
\chardef\ec@doubleacute@u="B7

% ring
\def\ec@ring#1{{\accent"6 #1}}
% \chardef\ec@ring@A="C5
% \chardef\ec@ring@a="E5
\chardef\ec@ring@U="97
\chardef\ec@ring@u="B7

% caron
\def\ec@caron#1{{\accent"7 #1}}
\chardef\ec@caron@C="83
\chardef\ec@caron@c="A3
\chardef\ec@caron@D="84
\chardef\ec@caron@d="A4
\chardef\ec@caron@E="85
\chardef\ec@caron@e="A5
\chardef\ec@caron@L="89
\chardef\ec@caron@l="A9
\chardef\ec@caron@N="8C
\chardef\ec@caron@n="AC
\chardef\ec@caron@R="90
\chardef\ec@caron@r="B0
\chardef\ec@caron@S="92
\chardef\ec@caron@s="B2
\chardef\ec@caron@T="94
\chardef\ec@caron@t="B4
\chardef\ec@caron@Z="9A
\chardef\ec@caron@z="BA

% breve
\def\ec@breve#1{{\accent"8 #1}}
\chardef\ec@breve@G="87
\chardef\ec@breve@g="A7

% macron
\def\ec@macron#1{{\accent"9 #1}}

% dot above
\def\ec@dotabove#1{{\accent"A #1}}
\chardef\ec@dotabove@Z="9B
\chardef\ec@dotabove@z="BB

% cedilla
\def\ec@cedilla#1{{\setbox\z@\hbox{#1}\ifdim\ht\z@=1ex\accent"0B #1%
  \else\ooalign{\unhbox\z@\crcr\hidewidth\char"0B\hidewidth}\fi}}
\chardef\ec@cedilla@C="C7
\chardef\ec@cedilla@c="E7
\chardef\ec@cedilla@S="93
\chardef\ec@cedilla@s="B3
\chardef\ec@cedilla@T="95
\chardef\ec@cedilla@t="B5
\def\ec@cedilla@g{\accent`\`g}

% ogonek
\def\ec@ogonek#1{{\ooalign{\null#1\crcr\hidewidth\char"0C\hidewidth}}}
\chardef\ec@ogonek@A="81
\chardef\ec@ogonek@a="A1
\chardef\ec@ogonek@E="86
\chardef\ec@ogonek@e="A6
%%% lowercase u is special
\def\ec@ogonek@u{{\ooalign{\null u\crcr\hidewidth\char"0C}}}

% bar under
\def\b#1{{\o@lign{\relax#1\crcr\hidewidth\sh@ft{-3ex}%
  \vbox to.2ex{\hbox{\char"09}\vss}\hidewidth}}}

%%% A special purpose macro
% catalan dot
\def\c@talandot#1{\kern#1em\llap{$\m@th\cdot$}\kern-#1em}
\def\Lmiddledot{L\c@talandot{-.1}}
\def\lmiddledot{l\c@talandot{.15}}

\catcode`@=12

\endinput
